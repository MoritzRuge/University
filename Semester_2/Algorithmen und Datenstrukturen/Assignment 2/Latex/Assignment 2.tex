\documentclass[a4paper]{assignment}

\usepackage[left=3cm, right=2.5cm]{geometry} % Hier die Ränder definieren

\usepackage [latin1]{inputenc}


\usepackage[pdftex]{graphicx}
%%%%% Path for Pictures %%%%%%
\graphicspath{ {./Pictures} }
\usepackage{subfigure} 
\usepackage[backend=biber,style=numeric,url=true]{biblatex} % für Quellenangaben und Bibliotheken
\usepackage[]{hyperref} % für hyperlinks
\usepackage{array} % für die Tabelle
\addbibresource{quellen.bib}  % Add your .bib file here

% Paket für die Baumdiagramdarstellung
\usepackage{tikz}
\usetikzlibrary{trees}

% https://chatgpt.com/c/68191c8e-8760-8002-843b-e82091955567
\usepackage{booktabs} % für schönere Tabellenlinien
\usepackage{amsmath}  % für mathematische Symbole wie \left[ \right]
\usepackage{amssymb}

%%%% Packet um die Farbe von text zu ändern %%%%
%\textcolor{red}{text}
%\definecolor{mygreen}{RGB}{0,150,0}
%\textcolor{mygreen}{Dies ist grüner Text mit eigener Farbe.}
%{\color{blue}
%Dies ist ein ganzer Absatz in Blau.
%}
\usepackage{xcolor}

%%%%% draws layout frames %%%%%
%\usepackage{showframe}

%%%%% Multiple Columns %%%%%
% see: https://www.overleaf.com/learn/latex/Multiple_columns
\usepackage{multicol}


% Text block indentation
%\lipsum[1]
%\begin{addmargin}[1em]{2em}% 1em left, 2em right
%\lipsum[2]
%\end{addmargin}
%\lipsum[3]
\usepackage{scrextend}



\hypersetup{
    colorlinks=true,
    linkcolor=blue,
    filecolor=magenta,      
    urlcolor=cyan,
    pdftitle={Overleaf Example},
    pdfpagemode=FullScreen,
    }
    

%%% Code-Block design
%%% Benutzung mit: 
%%%%% \lstinputlisting[language=Scala]{./Code/Hashtable.scala} %%% für datein
%%%%% \begin{lstlisting}[language=Scala] %%% für \begin und \end
% https://www.overleaf.com/learn/latex/Code_listing
\usepackage{listings}

\definecolor{codegreen}{rgb}{0,0.6,0}
\definecolor{codegray}{rgb}{0.5,0.5,0.5}
\definecolor{codepurple}{rgb}{0.58,0,0.82}
\definecolor{backcolour}{rgb}{0.95,0.95,0.92}

\lstdefinestyle{mystyle}{
    backgroundcolor=\color{backcolour},   
    commentstyle=\color{codegreen},
    keywordstyle=\color{magenta},
    numberstyle=\tiny\color{codegray},
    stringstyle=\color{codepurple},
    basicstyle=\ttfamily\footnotesize,
    breakatwhitespace=false,         
    breaklines=true,                 
    captionpos=b,                    
    keepspaces=true,                 
    numbers=left,                    
    numbersep=5pt,                  
    showspaces=false,                
    showstringspaces=false,
    showtabs=false,                  
    tabsize=2
}
\lstset{style=mystyle}


\begin{document}


\section*{Problem 1: Kryptographische Hashfunktionen und Blockchain}

\subsection*{a) Kryptographische Hashfunktionen in Scala}

Welche kryptographischen Hashfunktionen sind in Scala implementiert?  
Wie kann man sie verwenden?


\subsubsection*{Lösung: Non-cryptographic hashfunctions}

Scala3 hat keine eigene Kryptographische Hashfunktion Implementiert (jedenfalls habe ich nichts gefunden). \\

In Scala hat man die Möglichkeit interne Hashfunktionen zu benutzten wie z.B. mit hashCode() methode\cite{hashCode}:
\begin{lstlisting}[language=Scala]
scala> val result = "hello".hashCode()
val result: Int = 99162322
\end{lstlisting}
Dies dient aber nicht der Kryptographischen Verschlüsselung von Werten, da es hierbei zu viele Kollisionen kommt, eher ist es zur Kontrolle von Werten gedacht.\\

Eine weitere Möglichkeit ist es über eine zusätzliche Scala Library zusätzliche Hashfunktionen zu benutzen: \textit{scala.util.hashing.Hashing} \cite{scala.util}\\

\lstinputlisting[language=Scala]{./Code/Scalatest.scala}

\begin{verbatim}
Output: 69490486
\end{verbatim}

\subsubsection*{MurmurHash3}

Oder auch eine Implementierung von MurmurHash3 von Rex Kerr. Auch dieser ist aber ein \textit{non-cryptographic hashing algorithm} \cite{murmurhash3}

\lstinputlisting[language=Scala]{./Code/murmurHash3.scala}

\begin{verbatim}
Output: -608680269
\end{verbatim}

\subsubsection*{Kryptographische Hashfunktionen}

Um Kryptographische Hashfunktionen in Scala zu benutzen, müssen wir auf Bibliotheken von Java zurückgreifen. Um dies zu tun, Importieren wir z.B.:\textit{java.security.MessageDigest} - für die Nutzung von SHA-256, MD5 oder auch SHA-1.\cite{sha-256}

\begin{lstlisting}[language=Scala]
import java.security.MessageDigest
\end{lstlisting}

Um z.B.: SHA-256 zu verwenden, müssen wir die vorgefertigten Methoden, getInstance(), digest(), benutzen

\begin{lstlisting}[language=Scala]
import java.security.MessageDigest


@main def run(): Unit =

  val message = "Hello World"
  val sha256 = MessageDigest.getInstance("SHA-256")
  val hashWert = sha256.digest(message.getBytes("UTF-8"))

  println(hashWert)
\end{lstlisting}

\begin{verbatim}
Output: [B@45820e51
\end{verbatim}

\begin{itemize}
\item MessageDigest - ruft das Objekt auf
\item getInstance() - Returns a MessageDigest object that implements the specified digest algorithm.
\item digest - Performs a final update on the digest using the specified array of bytes, then completes the digest computation.
\end{itemize}

Da wir bei \textit{println(hashWert)} eine Standard-toString-Ausgabe von einem Java-Array erhalten( also in Bytes ), müssen wir die Ausgabe noch einmal in Hex-Zahlen umwandeln:

\lstinputlisting[language=Scala]{./Code/SHA-256.scala}
\begin{verbatim}
Output: a591a6d40bf420404a011733cfb7b190d62c65bf0bcda32b57b277d9ad9f146e
\end{verbatim}

\vspace{1em}

\subsection*{b) Verkettete Liste mit Hashreferenzen}

Implementieren Sie in \texttt{Scala} eine einfach verkettete Liste mit Hashreferenzen.  
In den Knoten der einfach verketteten Liste sollen \texttt{String}-Objekte gespeichert werden.  
Verwenden Sie dazu eine kryptographische Hashfunktion wie in Teil (a).

\vspace{1em}

\subsection*{c) Nonce und Hash mit Nullen am Ende}

Fügen Sie zu den Knoten Ihrer einfach verketteten Liste jeweils ein \emph{Nonce} hinzu,  
und stellen Sie sicher, dass die Hashwerte in den Referenzen alle mit acht Nullen (in der Binärdarstellung) enden.  

Wie viele Versuche sind dazu im Durchschnitt nötig?

\newpage



\section{Problem: Suchen in Zeichenketten II}

\subsection{Rabin-Karp mit mehreren Suchmustern}\label{Rabin-Karp-suchmuster}
Der Algorithmus von Rabin-Karp lässt sich leicht auf mehrere Suchmuster
verallgemeinern. Gegeben eine Zeichenkette $s$ und Suchmuster $t1 , \dots , tk$ , bestimme die erste Stelle in $s$, an der eines der Muster $t1, \dots , tk$ vorkommt. Beschreiben Sie, wie man den Algorithmus von Rabin-Karp für diese Situation anpassen kann. Was ist die heuristische Laufzeit Ihres Algorithmus (unterder Annahme, dass Kollisionen selten sind)?

\subsubsection{Probelmstellung:}
Gegeben:
\begin{itemize}
	\item Eine Zeichenkette $s$ (Text) der Länge $n$
	\item Ein Suchmuster $k$ mit $t_1, t_2, \dots , t_k$ der gleichen Länge $m$
\end{itemize}

\subsubsection{Gesucht:}
\begin{itemize}
	\item Ein Algorithmus: der die erste Position in $s$ (Text), an der irgendeins der Muster $t_1, \dots, t_k$ vorkommt.
	\item Die Laufzeit des Algorithmus (unter der Annahme, dass Kollisionen selten sind)
\end{itemize}

\subsubsection{Lösung:}
\subparagraph{Rabin-Karp vorgehen:}
\begin{itemize}
	\item Wir berechnen den Hashwert des Musters $t$
	\item Wir Iterieren über den Text $s$ mit einem Fenster/Bereich der Länge $m$
	\item Berechne den Hashwert des aktuellen Fensters $s[i \dots i+m-1]$
	\item Wenn die Hashwerte übereinstimmen, vergleichen wir den Text-ausschnitt und Muster direkt, um Kollisionen zu umgehen
\end{itemize}

\subparagraph{Rabin-Karb für mehrere Muster:}\cite{Stackoverflow-Rabin-Karp}
\begin{itemize}
	\item Wir berechnen den Hashwert für alle Suchmuster $t_1,t_2, \dots, t_k$
	\begin{itemize}
		\item Diese Hashwerte speichern wir in einer Datenstruktur (HashSets)
		\item Dadurch ist die Überprüfung, ob ein Hashwert zu einem Muster gehört, in $O(1)$ möglich
	\end{itemize}
	\item Wiederholen des normalen Algorithmus, Iterieren über alle Teilstrings der Länge $m$ im Text $s$
	\item Berechnen des aktuellen Hashwertes vom Fenster
	\item Vergleichen, ob dieser Hashwert in der Menge der HashSets zu finden ist
	\item Wenn die Hashwerte übereinstimmen, verlgeichen wir wieder direkt
\end{itemize}

\subparagraph{Eigenschaften eines HashSets in Scala:}

Eine HashSet-Struktur ist eine Datenstruktur, die eine Menge von eindeutigen Werten speichert und sehr schnelle Einfüge-, Such- und Löschoperationen erlaubt - im Schnitt in konstanter Zeit $o(1)$

\begin{itemize}
	\item HashSets haben die Eigentschaft keine Duplikate zu erlauben, d.h. jeder Wert wird nur einmal gespeichert, doppelte werden ignoriert
	\item Schnelle Suche von Werten (sofern keine Kollision)
	\item Ein HashSet verwendet intern eine Hashfunktion um die Werte zu speichern und zu finden
\end{itemize}

\noindent
Um HashSets zu benutzten importieren wir folgende Bibliothek:
\begin{lstlisting}[language=Scala]
import scala.collection.mutable.HashSet
\end{lstlisting}

\subparagraph{Pseudocode} könnte wie folgt aussehen:

\begin{lstlisting}[language=Scala]
import scala.collection.mutable.HashSet

// Hashfunktion mit Rolling Hash 
def hash(s: String): Int = ...

val musterHashes = HashSet[Int]() // Initialisiere HashSet
val muster = List("bob", "tim", "leo") // Suchmuster s[i ... i+m-1]
val m = muster.length // m = Laenge des Musters bei unterschiedlicher musterlaenge sollte m die wenigsten charaktere haben

// Rabin-Karb vorgehen fuer mehrere Muster:
// 1. Wir berechnen den Hashwert fuer alle Suchmuster t1,t2,...,tk
for muster <- muster do
	musterHashes.add(hash(muster)) //speicher die Hashwerte im Set
	
// wir Iterieren ueber den Text s mit der Fenstergroesse von m
for i <- 0 to s.lenght - m do
	val fenster = s.fenster(i, i + m)
	cal fensterHash = hash(fenster)
	
	if musterHashes.contains(fensterHash) then
		// Direkte kontrolle ob die muster(string) und Text uebereinstimmen
		if muster.contains(fenster) then
			return i
\end{lstlisting}

\subparagraph{Heuristische Laufzeit:}
\begin{itemize}
	\item Vorverarbeiten der Muster: $O(k)$
	\item Iteration über Textfenster: $O(n)$
	\item Hashvergleiche je Fenster: $O(1)$
	\item Unter annahme von seltenen Kollisionen, Vergleich: $O(1)$
\end{itemize}
$\Rightarrow$ Heuristische Gesamtlaufzeit: $O(n+k)$

\newpage
\subsection{Implimentierung des Algorithmus}
Implementieren Sie Ihren Algorithmus aus \ref{Rabin-Karp-suchmuster}. Beantworten Sie sodann folgende Frage: Was kommt öfter in dem Roman Sense \& Sensibility vor: sense oder sensibility/sensible?\\
Hinweis: Siehe http://www.gutenberg.org/files/161/161-0.txt.

\subsubsection{Implimentierung:}

\lstinputlisting[language=Scala]{./Code/Rabin-Karb.scala}

\begin{verbatim}
Output:
sense kommt 41 mal vor.
sensible kommt 20 mal vor.
Sensibility kommt 15 mal vor.
sensibility und sensible kommen 35 mal vor
sense kommt öfter vor
\end{verbatim}



\newpage

\pbitem{AVL-Bäume}
\begin{enumerate}
\item Fügen Sie die Schlüssel A, L, G, O, D, T, S, X, Y, Z in dieser Reihenfolge in
einen anfangs leeren AVL-Baum ein. Löschen Sie sodann die Schlüssel Z, A,
L. Zeichnen Sie den Baum nach jedem Einfüge- und Löschvorgang, und zeigen
Sie die Rotationen, welche durchgeführt werden. Annotieren Sie dabei auch
die Knoten mit ihrer jeweiligen Höhe.

\item Beweisen Sie: Beim Einfügen in einen AVL-Baum wird höchstens eine (Einfach-
oder Doppel-)Rotation ausgeführt. Gilt das auch beim Löschen (Begründung)?
\end{enumerate}


	
	
	
\end{problemlist}

%Quellen Reference
%\newpage
%\printbibliography

\end{document}