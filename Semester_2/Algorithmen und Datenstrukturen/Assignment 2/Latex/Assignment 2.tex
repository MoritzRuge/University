\documentclass[a4paper]{assignment}



\coursetitle{Algorithmen und Datenstrukturen}
\courselabel{AuD}
\exercisesheet{Assignment 1}{Algorithmen und Datenstrukturen}
\student{Moritz Ruge \& Lennard}
\school{Bachelor Informatik}
\university{Freie Universitäten Berlin}
\semester{Sommer Semester 2025}
\date{20 April 2025}
%\usepackage[pdftex]{graphicx}
%\usepackage{subfigure} 
\usepackage[backend=biber,style=numeric,url=true]{biblatex} % für Quellenangaben und Bibliotheken
\usepackage[]{hyperref} % für hyperlinks
\usepackage{array} % für die Tabelle
\usepackage{xcolor} %für Farben bei Text
\addbibresource{ref.bib}  % Add your .bib file here

% Paket für die Baumdiagramdarstellung
\usepackage{tikz}
\usetikzlibrary{trees}


\hypersetup{
    colorlinks=true,
    linkcolor=blue,
    filecolor=magenta,      
    urlcolor=cyan,
    pdftitle={Overleaf Example},
    pdfpagemode=FullScreen,
    }

\begin{document}

% Aufgabe 1 Binäre Suchbäume)
\begin{problemlist}

\pbitem{Induktion und binäre Bäume}
Ein binärer Baum heißt vollständig, falls jeder Knoten entweder null oder zwei Kinder besitzt.
\begin{enumerate}
\item Zeichnen Sie einen binären Suchbaum, der vollständig ist, und einen binären Suchbaum, der nicht vollständig ist.

\item Beweisen Sie durch eine geeignete Induktion: In jedem vollständigen binären Suchbaum ist die Anzahl der Blätter genau um eins größer als die Anzahl der inneren Knoten.

\item Formulieren Sie eine ähnliche Aussage für allgemeine binäre Suchbäume und beweisen Sie sie.
\end{enumerate}


\pbitem{Binäre Suchbäume}


\pbitem{AVL-Bäume}



	
	
	
\end{problemlist}

%Quellen Reference
%\newpage
%\printbibliography

\end{document}