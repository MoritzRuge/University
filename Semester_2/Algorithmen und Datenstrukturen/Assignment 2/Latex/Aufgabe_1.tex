\section*{Aufgabe 1 – Induktion und binäre Bäume}

\subsection*{b) Aussage für vollständige binäre Suchbäume}

In jedem vollständigen binären Suchbaum gilt: Jeder innere Knoten hat genau zwei Kindknoten.

\textbf{Definitionen:}
\begin{itemize}
    \item Ein \textbf{innerer Knoten} ist ein Knoten $v$, der mindestens einen Kindknoten besitzt.
    \item Ein \textbf{Blattknoten} ist ein Knoten $v$, der keine Kindknoten besitzt.
\end{itemize}

\textbf{Behauptung:} In jedem vollständigen binären Suchbaum gilt:
\[
\text{Anzahl der Blätter} = \text{Anzahl der inneren Knoten} + 1
\]

\textbf{Beweis durch Induktion:}

\textbf{Induktionsanfang (IA):} Ein vollständiger binärer Baum mit genau einem inneren Knoten:

\[
\begin{tikzpicture}[scale=0.6, level distance=1.2cm, sibling distance=2cm]
\node{a}
	child {node {b}}
	child {node {c}};
\end{tikzpicture}
\]

Ein innerer Knoten (a), zwei Blattknoten (b, c). Die Aussage gilt: $2 = 1 + 1$.

\textbf{Induktionsvoraussetzung (IV):} Die Aussage gelte für einen vollständigen Baum mit $n$ inneren Knoten:
\[
\text{Blätter} = n + 1
\]

\textbf{Induktionsschritt (IS):} Füge einen weiteren inneren Knoten hinzu. Dies kann nur durch Umwandlung eines bisherigen Blattes erfolgen, das dann zwei neue Kinder erhält. Dadurch:
\begin{itemize}
    \item Wird 1 Blatt zu einem inneren Knoten $\Rightarrow$ $-1$ Blatt
    \item Zwei neue Blätter entstehen $\Rightarrow$ $+2$ Blätter
\end{itemize}

Gesamtänderung: $+1$ innerer Knoten, $+1$ Blatt. Damit gilt auch für $n+1$:
\[
\text{Blätter} = (n+1) + 1
\]

\subsection*{c) Allgemeine Aussage für beliebige binäre Bäume}

\textbf{Behauptung:} Für jeden Baum gilt:
\[
\text{Anzahl der Knoten} = \text{Anzahl der Kanten} + 1
\]

\textbf{Begründung:}
\begin{itemize}
    \item Jeder Knoten (außer der Wurzel) hat genau eine eingehende Kante vom Elternknoten.
    \item Beim Einfügen eines neuen Knotens entsteht genau eine neue Kante.
\end{itemize}

\textbf{Induktionsanfang:}
Ein einzelner Knoten (z.B: "a") hat 0 Kanten, 1 Knoten. Gilt: $1 = 0 + 1$.

\textbf{Induktionsvoraussetzung:} Für einen Baum mit $n$ Knoten gilt:
\[
n = \text{Kantenanzahl} + 1
\]

\textbf{Induktionsschritt:}
Ein neuer Knoten wird eingefügt:
\begin{itemize}
    \item Eine neue Kante entsteht.
    \item Knotenanzahl wird $n+1$
\end{itemize}
Dann gilt:
\[
n + 1 = (\text{Kantenanzahl} + 1) + 1 - 1 = \text{Kantenanzahl}_{neu} + 1
\]
$\Box$


