\section*{Problem 1: Kryptographische Hashfunktionen und Blockchain}

\subsection*{a) Kryptographische Hashfunktionen in Scala}

Welche kryptographischen Hashfunktionen sind in Scala implementiert?  
Wie kann man sie verwenden?


\subsubsection*{Lösung: Non-cryptographic hashfunctions}

Scala3 hat keine eigene Kryptographische Hashfunktion Implementiert (jedenfalls habe ich nichts gefunden). \\

In Scala hat man die Möglichkeit interne Hashfunktionen zu benutzten wie z.B. mit hashCode() methode\cite{hashCode}:
\begin{lstlisting}[language=Scala]
scala> val result = "hello".hashCode()
val result: Int = 99162322
\end{lstlisting}
Dies dient aber nicht der Kryptographischen Verschlüsselung von Werten, da es hierbei zu viele Kollisionen kommt, eher ist es zur Kontrolle von Werten gedacht.\\

Eine weitere Möglichkeit ist es über eine zusätzliche Scala Library zusätzliche Hashfunktionen zu benutzen: \textit{scala.util.hashing.Hashing} \cite{scala.util}\\

\lstinputlisting[language=Scala]{./Code/Scalatest.scala}

\begin{verbatim}
Output: 69490486
\end{verbatim}

\subsubsection*{MurmurHash3}

Oder auch eine Implementierung von MurmurHash3 von Rex Kerr. Auch dieser ist aber ein \textit{non-cryptographic hashing algorithm} \cite{murmurhash3}

\lstinputlisting[language=Scala]{./Code/murmurHash3.scala}

\begin{verbatim}
Output: -608680269
\end{verbatim}

\subsubsection*{Kryptographische Hashfunktionen}

Um Kryptographische Hashfunktionen in Scala zu benutzen, müssen wir auf Bibliotheken von Java zurückgreifen. Um dies zu tun, Importieren wir z.B.:\textit{java.security.MessageDigest} - für die Nutzung von SHA-256, MD5 oder auch SHA-1.\cite{sha-256}

\begin{lstlisting}[language=Scala]
import java.security.MessageDigest
\end{lstlisting}

Um z.B.: SHA-256 zu verwenden, müssen wir die vorgefertigten Methoden, getInstance(), digest(), benutzen

\begin{lstlisting}[language=Scala]
import java.security.MessageDigest


@main def run(): Unit =

  val message = "Hello World"
  val sha256 = MessageDigest.getInstance("SHA-256")
  val hashWert = sha256.digest(message.getBytes("UTF-8"))

  println(hashWert)
\end{lstlisting}

\begin{verbatim}
Output: [B@45820e51
\end{verbatim}

\begin{itemize}
\item MessageDigest - ruft das Objekt auf
\item getInstance() - Returns a MessageDigest object that implements the specified digest algorithm.
\item digest - Performs a final update on the digest using the specified array of bytes, then completes the digest computation.
\end{itemize}

Da wir bei \textit{println(hashWert)} eine Standard-toString-Ausgabe von einem Java-Array erhalten( also in Bytes ), müssen wir die Ausgabe noch einmal in Hex-Zahlen umwandeln:

\lstinputlisting[language=Scala]{./Code/SHA-256.scala}
\begin{verbatim}
Output: a591a6d40bf420404a011733cfb7b190d62c65bf0bcda32b57b277d9ad9f146e
\end{verbatim}

\vspace{1em}

\newpage
\subsection*{b) Verkettete Liste mit Hashreferenzen}

Implementieren Sie in \texttt{Scala} eine einfach verkettete Liste mit Hashreferenzen.  
In den Knoten der einfach verketteten Liste sollen \texttt{String}-Objekte gespeichert werden.  
Verwenden Sie dazu eine kryptographische Hashfunktion wie in Teil (a).

\lstinputlisting[language=Scala]{./Code/1b.scala}

\begin{verbatim}
Output:
Daten: Alice
PrevHash: 5923931d867e648ec3e488074d631134d596b6a5424c5165258e9a6475fdc777

Daten: Bob
PrevHash: 12775e79fe15fd6aa0fcf6605550b6cc45ec10552c6d0b72685815af763f4774

Daten: Charlie
PrevHash: 0000000000000000000000000000000000000000000000000000000000000000
\end{verbatim}

\newpage
\subsection*{c) Nonce und Hash mit Nullen am Ende}

Fügen Sie zu den Knoten Ihrer einfach verketteten Liste jeweils ein \emph{Nonce} hinzu,  
und stellen Sie sicher, dass die Hashwerte in den Referenzen alle mit acht Nullen (in der Binärdarstellung) enden.  

Wie viele Versuche sind dazu im Durchschnitt nötig?\\

\lstinputlisting[language=Scala]{./Code/1c.scala}

\begin{verbatim}
Output:
Daten: Alice
PrevHash: 8efdfdc4cc1b5cb65ea1570e1371eac8ae7fb84f1b65c4f2b6d7f5d85f7c5700
Nonce:    471
Hash:     53c7e0c85a3182285cf4d837dfa1d495d5c883ac94f43b970c567b4d0a314e00

Daten: Bob
PrevHash: 213f013d20bc293b36a9fa7a59444aed5a881af4f17a41de31ca27ec7cf9e100
Nonce:    101
Hash:     8efdfdc4cc1b5cb65ea1570e1371eac8ae7fb84f1b65c4f2b6d7f5d85f7c5700

Daten: Charlie
PrevHash: 0000000000000000000000000000000000000000000000000000000000000000
Nonce:    47
Hash:     213f013d20bc293b36a9fa7a59444aed5a881af4f17a41de31ca27ec7cf9e100
\end{verbatim}

Die Ausgabe des hashes erfolgt in Hexadezimal Darstellung.\\
Ein Byte = 8 Bit = 00000000 bis 11111111 $\rightarrow$ was in Hex = \textbf{00} bis FF sind.  Das heißt, die letzten beiden Nullen sind die Hexadezimaldarstellung für 8 Nullen in Binärdarstellung.

\subsubsection*{Druchschnittliche Versuche:}

Die Kryptographische Hashfunktion SHA-256 produziert genau 256 Bit oder auch 32 Byte. Die Wahrscheinlichkeit das ein Bit entweder 0 oder 1 annimmt ist $\frac{1}{2}$. Da die Letzten 8 Bit oder auch 1 Byte Null sein sollen ergibt sich folgende Formel:\\

\begin{center}
$P(alle 8 Bits = 0)=(\frac{1}{2})^8 = \frac{1}{256}$
\end{center}

Der Erwartungswert ist dann:

\begin{center}
$\text{Erwartungswert}=\frac{1}{P}= \frac{1}{\frac{1}{256}} = 256$
\end{center}

Die Wahrscheinlichkeit um am Ende Acht Nullen zu ziehen ist also $\frac{1}{256}$, bzw. braucht man ungefähr 256 versuche um den gewünschten Nonce zu berechnen.