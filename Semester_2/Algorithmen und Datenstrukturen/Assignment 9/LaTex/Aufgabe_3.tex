\section{Problem: Suche in Zeichenketten III}

 Sei $\sum = C, G, T, A.$ Sei $s = CTTGGATTA$ und $t = TTA$.



\subsection{Naiver Algorithmus}
Verwenden Sie den naiven Algorithmus, um festzustellen, ob/wo das Muster t
in der Zeichenkette s vorkommt.
Zeigen Sie die einzelnen Schritte.

\vspace{1em}

\begin{verbatim}
1. i = 0, j = 0 --> s[0], t[0]: C != T
2. i = 1, j = 0 --> s[1], t[0]: T == T
2.2. i = 2, j = 1 --> s[2], t[1]: T == T
2.3. i = 3, j = 2 --> s[3], t[2]: G != T
3. i = 4, j = 0 --> s[4], t[0]: G != T
4. i = 5, j = 0 --> s[5], t[0]: A != T
5. i = 6, j = 0 --> s[6], t[0]: T == T
5.2. i = 7, j = 1 --> s[7], t[1]: T == T
5.3. i = 8, j = 2 --> s[8], t[2]: A == A ==> t found.
Der String t kommt am Index 6/ 7. Position in s vor.

A : 0, T : 1, G : 2, C : 3 und die Primzahl 5
\end{verbatim}

\subsection{Rabin-Karp Algorithmus}
Verwenden Sie den Algorithmus von Rabin-Karp, um festzustellen, ob/wo das
Muster t in der Zeichenkette s vorkommt. Verwenden Sie A : 0, T : 1, G : 2, C :
3 und die Primzahl 5 als Modulus für die Hashfunktion.
Zeigen Sie die einzelnen Schritte.

\vspace{1em}

\begin{verbatim}
s = CTTGGATTA und t = TTA
h(t) = h(TTA) = 4^2+4^1+0 = 4^2+4 = 20 mod 5 = 0
1. h(CTT) = 3*4^2+4^1+4^0 = 56 mod 5 = 1 --> 0 != 1 -> not found
2. h(TTG) = 4^2+4+2 = 22 mod 5 = 2 --> 0 != 2 -> not found
3. h(TGG) = 4^2+2*4+2 = 26 mod 5 = 1 --> 0 != 1 -> not found
4. h(GGA) = 2*4^2+2*4 = 40 mod 5 = 0 --> 0 == 0 -> found at position 4 / index 3
Charaktervergleich nach dem naiven Algorithmus:
1. i = 0, j = 0 --> s'[0], t[0]: G != T -> not found t ist nicht an der Position 4 n s.
5. h(GAT) = 2*16+1 = 17 mod 5 = 2 --> 0 != 2 -> not found
6. h(ATT) = 4+1 = 5 mod 5 = 0 --> 0 == 0 -> found at position 6 / index 5
Charaktervergleich nach dem naiven Algorithmus:
1. i = 0, j = 0 --> s'[0], t[0]: A != T -> not found t ist nicht an der Position 6 n s.
7. h(TTA) = 16+4 = 20 mod 5 = 0 --> 0 == 0 found at position 6/ Index 7
Charaktervergleich nach dem naiven Algorithmus:
1. i = 0, j = 0 --> s'[0], t[0]: T == T
2. i = 1, j = 1 --> s'[1], t[1]: T == T
3. i = 2, j = 2 --> s'[2], t[2]: A == A
s' == t
Der String t kommt am Index 6/ 7. Position in s vor.
\end{verbatim}

\subsection{Knuth-Morris-Pratt Algorithmus}
Verwenden Sie den Algorithmus von Knuth-Morris-Pratt, um festzustellen,
ob/wo das Muster t in der Zeichenkette s vorkommt.
Zeigen Sie die einzelnen Schritte.

\alert{TODO}
