\section{Problem: Suchen in Zeichenketten II}

\subsection{Rabin-Karp mit mehreren Suchmustern}\label{Rabin-Karp-suchmuster}
Der Algorithmus von Rabin-Karp lässt sich leicht auf mehrere Suchmuster
verallgemeinern. Gegeben eine Zeichenkette $s$ und Suchmuster $t1 , \dots , tk$ , bestimme die erste Stelle in $s$, an der eines der Muster $t1, \dots , tk$ vorkommt. Beschreiben Sie, wie man den Algorithmus von Rabin-Karp für diese Situation anpassen kann. Was ist die heuristische Laufzeit Ihres Algorithmus (unterder Annahme, dass Kollisionen selten sind)?

\subsubsection{Probelmstellung:}
Gegeben:
\begin{itemize}
	\item Eine Zeichenkette $s$ (Text) der Länge $n$
	\item Ein Suchmuster $k$ mit $t_1, t_2, \dots , t_k$ der gleichen Länge $m$
\end{itemize}

\subsubsection{Gesucht:}
\begin{itemize}
	\item Ein Algorithmus: der die erste Position in $s$ (Text), an der irgendeins der Muster $t_1, \dots, t_k$ vorkommt.
	\item Die Laufzeit des Algorithmus (unter der Annahme, dass Kollisionen selten sind)
\end{itemize}

\subsubsection{Lösung:}
\paragraph{Rabin-Karp vorgehen:}
\begin{itemize}
	\item Wir berechnen den Hashwert des Musters $t$
	\item Wir Iterieren über den Text $s$ mit einem Fenster/Bereich der Länge $m$
	\item Berechne den Hashwert des aktuellen Fensters $s[i \dots i+m-1]$
	\item Wenn die Hashwerte übereinstimmen, vergleichen wir den Text-ausschnitt und Muster direkt, um Kollisionen zu umgehen
\end{itemize}

\paragraph{Rabin-Karb für mehrere Muster:}\cite{Stackoverflow-Rabin-Karp}
\begin{itemize}
	\item Wir berechnen den Hashwert für alle Suchmuster $t_1,t_2, \dots, t_k$
	\begin{itemize}
		\item Diese Hashwerte speichern wir in einer Datenstruktur (HashSets)
		\item Dadurch ist die Überprüfung, ob ein Hashwert zu einem Muster gehört, in $O(1)$ möglich
	\end{itemize}
	\item Wiederholen des normalen Algorithmus, Iterieren über alle Teilstrings der Länge $m$ im Text $s$
	\item Berechnen des aktuellen Hashwertes vom Fenster
	\item Vergleichen, ob dieser Hashwert in der Menge der HashSets zu finden ist
	\item Wenn die Hashwerte übereinstimmen, verlgeichen wir wieder direkt
\end{itemize}



\subsection{Implimentierung des Algorithmus}
Implementieren Sie Ihren Algorithmus aus \ref{Rabin-Karp-suchmuster}. Beantworten Sie sodann folgende Frage: Was kommt öfter in dem Roman Sense \& Sensibility vor: sense oder sensibility/sensible?\\
Hinweis: Siehe http://www.gutenberg.org/files/161/161-0.txt.


