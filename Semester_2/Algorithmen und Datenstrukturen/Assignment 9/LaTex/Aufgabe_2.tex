\section{Problem: Suchen in Zeichenketten II}

\subsection{Rabin-Karp mit mehreren Suchmustern}\label{Rabin-Karp-suchmuster}
Der Algorithmus von Rabin-Karp lässt sich leicht auf mehrere Suchmuster
verallgemeinern. Gegeben eine Zeichenkette $s$ und Suchmuster $t1 , \dots , tk$ , bestimme die erste Stelle in $s$, an der eines der Muster $t1, \dots , tk$ vorkommt. Beschreiben Sie, wie man den Algorithmus von Rabin-Karp für diese Situation anpassen kann. Was ist die heuristische Laufzeit Ihres Algorithmus (unterder Annahme, dass Kollisionen selten sind)?

\subsubsection{Probelmstellung:}
Gegeben:
\begin{itemize}
	\item Eine Zeichenkette $s$ (Text) der Länge $n$
	\item Ein Suchmuster $k$ mit $t_1, t_2, \dots , t_k$ der gleichen Länge $m$
\end{itemize}

\subsubsection{Gesucht:}
\begin{itemize}
	\item Ein Algorithmus: der die erste Position in $s$ (Text), an der irgendeins der Muster $t_1, \dots, t_k$ vorkommt.
	\item Die Laufzeit des Algorithmus (unter der Annahme, dass Kollisionen selten sind)
\end{itemize}

\subsubsection{Lösung:}
\subparagraph{Rabin-Karp vorgehen:}
\begin{itemize}
	\item Wir berechnen den Hashwert des Musters $t$
	\item Wir Iterieren über den Text $s$ mit einem Fenster/Bereich der Länge $m$
	\item Berechne den Hashwert des aktuellen Fensters $s[i \dots i+m-1]$
	\item Wenn die Hashwerte übereinstimmen, vergleichen wir den Text-ausschnitt und Muster direkt, um Kollisionen zu umgehen
\end{itemize}

\subparagraph{Rabin-Karb für mehrere Muster:}\cite{Stackoverflow-Rabin-Karp}
\begin{itemize}
	\item Wir berechnen den Hashwert für alle Suchmuster $t_1,t_2, \dots, t_k$
	\begin{itemize}
		\item Diese Hashwerte speichern wir in einer Datenstruktur (HashSets)
		\item Dadurch ist die Überprüfung, ob ein Hashwert zu einem Muster gehört, in $O(1)$ möglich
	\end{itemize}
	\item Wiederholen des normalen Algorithmus, Iterieren über alle Teilstrings der Länge $m$ im Text $s$
	\item Berechnen des aktuellen Hashwertes vom Fenster
	\item Vergleichen, ob dieser Hashwert in der Menge der HashSets zu finden ist
	\item Wenn die Hashwerte übereinstimmen, verlgeichen wir wieder direkt
\end{itemize}

\subparagraph{Eigenschaften eines HashSets in Scala:}

Eine HashSet-Struktur ist eine Datenstruktur, die eine Menge von eindeutigen Werten speichert und sehr schnelle Einfüge-, Such- und Löschoperationen erlaubt - im Schnitt in konstanter Zeit $o(1)$

\begin{itemize}
	\item HashSets haben die Eigentschaft keine Duplikate zu erlauben, d.h. jeder Wert wird nur einmal gespeichert, doppelte werden ignoriert
	\item Schnelle Suche von Werten (sofern keine Kollision)
	\item Ein HashSet verwendet intern eine Hashfunktion um die Werte zu speichern und zu finden
\end{itemize}

\noindent
Um HashSets zu benutzten importieren wir folgende Bibliothek:
\begin{lstlisting}[language=Scala]
import scala.collection.mutable.HashSet
\end{lstlisting}

\subparagraph{Pseudocode} könnte wie folgt aussehen:

\begin{lstlisting}[language=Scala]
import scala.collection.mutable.HashSet

// Hashfunktion mit Rolling Hash 
def hash(s: String): Int = ...

val musterHashes = HashSet[Int]() // Initialisiere HashSet
val muster = List("bob", "tim", "leo") // Suchmuster s[i ... i+m-1]
val m = muster.length // m = Laenge des Musters bei unterschiedlicher musterlaenge sollte m die wenigsten charaktere haben

// Rabin-Karb vorgehen fuer mehrere Muster:
// 1. Wir berechnen den Hashwert fuer alle Suchmuster t1,t2,...,tk
for muster <- muster do
	musterHashes.add(hash(muster)) //speicher die Hashwerte im Set
	
// wir Iterieren ueber den Text s mit der Fenstergroesse von m
for i <- 0 to s.lenght - m do
	val fenster = s.fenster(i, i + m)
	cal fensterHash = hash(fenster)
	
	if musterHashes.contains(fensterHash) then
		// Direkte kontrolle ob die muster(string) und Text uebereinstimmen
		if muster.contains(fenster) then
			return i
\end{lstlisting}

\subparagraph{Heuristische Laufzeit:}\alert{TODO}



\subsection{Implimentierung des Algorithmus}
Implementieren Sie Ihren Algorithmus aus \ref{Rabin-Karp-suchmuster}. Beantworten Sie sodann folgende Frage: Was kommt öfter in dem Roman Sense \& Sensibility vor: sense oder sensibility/sensible?\\
Hinweis: Siehe http://www.gutenberg.org/files/161/161-0.txt.


