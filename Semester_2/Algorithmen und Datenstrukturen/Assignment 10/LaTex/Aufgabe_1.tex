\section{Problem: Dynamisches Programmieren}

Sei s eine Zeichenkette der Länge n. Sie vermuten, dass es sich bei s um einen
deutschsprachigen Text handelt, bei dem die Leer- und Satzzeichen verloren gegan-
gen sind (also zum Beispiel s = “werreitetsospaetdurchnachtundwind”), und Sie
möchten den ursprünglichen Text rekonstruieren.

\vspace{1em}

\noindent
Dazu steht Ihnen ein Wörterbuch zur Verfügung, das in Form einer Funktion
$$dict : String \rightarrow Boolean$$

\noindent
implementiert ist. dict(w) liefert true für ein gültiges Wort w, und false sonst
(z.B. dict(“blau”) = true und dict(“bsau”) = false).
Verwenden Sie dynamisches Programmieren, um einen schnellen Algorithmus zu
entwickeln, der entscheidet, ob sich s als eine Aneinanderreihung von gültigen
Wörtern darstellen lässt. Gehen Sie dabei folgendermaßen vor:

\begin{enumerate}
	\item Definieren Sie geeignete Teilprobleme und geben Sie eine geeignete Rekursion an. Erklären Sie Ihre Rekursion in einem Satz.
	\item Geben Sie Pseudocode für Ihren Algorithmus an.
	\item Analysieren Sie die Laufzeit und Speicherplatzbedarf Ihres Algorithmus unter der Annahme, dass ein Aufruf von dict konstante Zeit benötigt.
	\item Beschreiben Sie in einem Satz, wie man eine gültige Wortfolge finden kann,
	falls sie existiert.
\end{enumerate}