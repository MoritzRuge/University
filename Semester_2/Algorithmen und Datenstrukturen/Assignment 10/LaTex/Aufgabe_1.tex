\section{Problem: Dynamisches Programmieren}

Sei s eine Zeichenkette der Länge n. Sie vermuten, dass es sich bei s um einen
deutschsprachigen Text handelt, bei dem die Leer- und Satzzeichen verloren gegan-
gen sind (also zum Beispiel s = “werreitetsospaetdurchnachtundwind”), und Sie
möchten den ursprünglichen Text rekonstruieren.

\vspace{1em}

\noindent
Dazu steht Ihnen ein Wörterbuch zur Verfügung, das in Form einer Funktion
$$dict : String \rightarrow Boolean$$

\noindent
implementiert ist. dict(w) liefert true für ein gültiges Wort w, und false sonst
(z.B. dict(“blau”) = true und dict(“bsau”) = false).
Verwenden Sie dynamisches Programmieren, um einen schnellen Algorithmus zu
entwickeln, der entscheidet, ob sich s als eine Aneinanderreihung von gültigen
Wörtern darstellen lässt. Gehen Sie dabei folgendermaßen vor:

\begin{enumerate}
	\item Definieren Sie geeignete Teilprobleme und geben Sie eine geeignete Rekursion an. Erklären Sie Ihre Rekursion in einem Satz.
	\item Geben Sie Pseudocode für Ihren Algorithmus an.
	\item Analysieren Sie die Laufzeit und Speicherplatzbedarf Ihres Algorithmus unter der Annahme, dass ein Aufruf von dict konstante Zeit benötigt.
	\item Beschreiben Sie in einem Satz, wie man eine gültige Wortfolge finden kann,
	falls sie existiert.
\end{enumerate}

\subsection{Teilproblem Definieren und Rekursion}

\paragraph{Teilproblem:} Betrachte dp[i] als boolsche Tabelle, wobei dp[i] = true, wenn das Präfix der Länge i in gültige Wörter aufgeteilt werden kann.  

\paragraph{Rekursion:} Ein Präfix $s[0 \dots i]$ bis Index i ist gültig (dp[i] = true), wenn es einen Index $j<i$ gibt, so dass: 
\begin{enumerate}
	\item Das Präfix bis j gültig ist (dp[j] = true)
	\item Die Zeichenkette von j zu i ein gültiges Wort bildet (durch Aufruf der Funktion dict)
\end{enumerate} 


\subsection{Pseudocode:}
\begin{lstlisting}[language=Scala]
def funktion(s: String, dict: String => Boolean): Boolean =
	val n = s.length
	Rval dp = Array.fill(n+1)(flase) // erzeugt Array mit False als default Value
	dp(0) = ture // leere Zeichenkette ist guetig
	
	for i <- 1 to n do
		for j <- 0 until i do
			if dp(j) && dict(s.substring(j, i)) then
				dp(i) = true
				
	dp(n)
\end{lstlisting}

\subsection{Laufzeitanalyse}

\begin{itemize}
	\item \textbf{Laufzeit:}\\
	Die Äußere Schleife läuft von 1 bis n, also n mal, die innere Schleife bis i Mal $\rightarrow$ also insgesamt:
	$$\sum_{i=1}^{n}i = O(n^2)$$
	
	\item \textbf{Speicherplatz:}\\ 
	Wie haben ein Array dp der Länge n+1 $\rightarrow$ also braucht der Speicher  $O(n)$
\end{itemize}

\subsection{Wortfolge}

Man merkt sich bei jeder erfolgreichen Trennung die Position $j$ und rekonstruiert die Wortgrenzen rückwärts von $dp[n]$ aus, um die Wortfolge zu erhalten.