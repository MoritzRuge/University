\section{Problem: Finden von Senken in Graphen}

Betrachtet man die Adjazenzmatrixdarstellung eines Graphen $G = (V, E)$, dann
haben viele Algorithmen Laufzeit $|V|2$ . Es gibt aber Ausnahmen. Zeigen Sie, dass
die Frage, ob ein gerichteter Graph $G$ eine globale Senke — einen Knoten vom
Eingrad $|V | - 1$ und Ausgrad 0 $-$ hat, in Zeit $O(|V|)$ beantwortet werden kann,
selbst wenn man die Adjazenzmatrixdarstellung von $G$ (die ja selbst schon die Größe
$O(|V|2$ ) hat) verwendet. Beweisen Sie Korrektheit und Laufzeit Ihres Algorithmus.

Hinweis: Sei A die Adjazenzmatrix von G und $u, v \in V , u \neq v$. Was folgt über u
und v, wenn $A_{uv} = 1$ ist? Was, wenn $A_{uv} = 0$ ist?

\subsection{Beispiel: Dreieck}

\begin{center}
\begin{tabular}{|c|c|c|c|c|}
	\hline
	& A & B & C & D \\
	\hline
	A & 0 & 1 & 0 & 1 \\
	\hline
	B & 0 & 0 & 1 & 1 \\
	\hline
	C & 1 & 0 & 0 & 1 \\
	\hline
	D & 0 & 0 & 0 & 0 \\
	\hline
\end{tabular}
\end{center}

\noindent
Ein Knoten kann in diesem Fall keine Kante zu sich selbst haben (keine Kreise)\\
Darüber hinaus, kann es maximal nur eine globale Senke geben. Dies ergibt sich aus der Definition von globalen Senken.

directed Graph $matrix[v][*]$ row outgoing edges, $matrix[*][v]$ coloumn incoming edges. $\rightarrow$ we care about $matrix[*][v]$

\begin{lstlisting}[language=Python]
i = 0 # row
j = 0 # cols
while i < len(matrix) and j < len(matrix): 	# solange i und j kleiner gleich der Anzahl von Knoten in der Matrix sind.
	if matrix[i][j] == 0:			# wenn der Knoten i keine ausgehende Kante zum Knoten j hat gehe nach rechts in der Matrix.		   
		i += 1
	else:					# Der Knoten i hat eine ausgehende Kante, kann also keine globale Senke sein. 
		j += 1		
gol_snk = i					# moegliche globale Senke Befindet sich am Index i bzw. j, da diese in einer quadratischen Matrix, auf den identischen Knoten zeigen. Im ersten Squentiellen Loop werden alle ausser einem Knoten als moegliche globale Senke ausgeschlossen. Ist ein matrix[i][j] == 1, ist der Knoten matrix[i] als Kandidat ausgeschlossen, da dieser eine ausgehende Kante hat. Die Suche wird fuer den naechsten Knoten fortgesetzt. Der Knoten i, bei dem die Schleifen bedigung i < len(matrix) and j < len(matrix) verletzt wird ist der Kandidate fuer die globale Senke. Dies geschiet nur wenn matrix[i] nur aus 0 besteht oder wenn i = len(matrix)-1 und beim unten weiter suchen matrix[i][j] die erste 0 gefunden wird.

# Ueberpruefe erneut, ob matrix[gol_snk] keine ausgehenden Kanten hat
for k in range(len(matrix)):
	if matrix[candidate][k] == 1: # es gibt doch eine ausgehende Kante -> es gibt keine globale Senke. 
		return -1  # std error code

# Ueberpruefe ob der Kandidat (matrix[*][gol_snk]) wirklich eine eingehende Kanten von jedem anderen Knoten hat, ausser von sich selbst.
for k in range(len(matrix)):
	if k != gol_snk and matrix[k][gol_snk] == 0: # alle Eintraege der Kandidatenspalte (ausser von sich selbst) muessen 1 sein. 
		return -1  # Not a sink

return candidate  # wenn alle Ueberpruefungen durchlaufen wurden, und nicht vorher abgebrochen wurde, wurde die globale Senke gefunden.
\end{lstlisting}

\subsection{Laufzeit}

\paragraph{Laufzeitkomplexität} im worst-case: $O(2|V|-1 +|V| + |V|) \rightarrow$ dominate Klasse ist $O(|V|)$\\
Die Überprüfungen dauern: $|V|$\\
Die erste Suche kann $2|V|-1$ dauern, da man $|V|-1$ mal nach rechts und unten gehen kann. Es ist also mögliche die globale Senke in linearer Zeit zufinden. $O(|V|)$ worst-case: $O(n)$\\ 
Die globale Senke garantiert und korrekt in nur $|V|$ Iterationen / Schritten zu finden ist nicht möglich.
