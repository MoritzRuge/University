\section{Problem: Finden von Senken in Graphen}

Betrachtet man die Adjazenzmatrixdarstellung eines Graphen $G = (V, E)$, dann
haben viele Algorithmen Laufzeit $|V|2$ . Es gibt aber Ausnahmen. Zeigen Sie, dass
die Frage, ob ein gerichteter Graph $G$ eine globale Senke — einen Knoten vom
Eingrad $|V | - 1$ und Ausgrad 0 $-$ hat, in Zeit $O(|V|)$ beantwortet werden kann,
selbst wenn man die Adjazenzmatrixdarstellung von $G$ (die ja selbst schon die Größe
$O(|V|2$ ) hat) verwendet. Beweisen Sie Korrektheit und Laufzeit Ihres Algorithmus.

Hinweis: Sei A die Adjazenzmatrix von G und $u, v \in V , u \neq v$. Was folgt über u
und v, wenn $A_{uv} = 1$ ist? Was, wenn $A_{uv} = 0$ ist?