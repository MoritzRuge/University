\section*{{Problem 2: Analyse von Skiplisten}} 

Sei $L$ eine Skipliste mit $n$ Einträgen.\\

\textbf{a)} Zeigen Sie, dass $L$ im Erwartungswert $O(n)$ Knoten besitzt.\\

\noindent
\textbf{Gegeben:}\\

\begin{center}
\begin{tabular}{ c c c c c c c c c c c }
 L0: & 1 & 2 & 3 & 4 & 5 & 6 & 7 & 8 & 9 & 10 \\ 
 L1: & 1 &  & 3 &  & 5 &  & 7 &  & 9 &  \\   
 L2: & 1 &  &  &  & 5 &  &  &  & 9 &  \\ 
 L3: &  &  &  &  & 5 &  &  &  &  &      
\end{tabular}
\end{center}
$\Rightarrow$ $L_i \equiv O(\log n)$\\
$\Rightarrow$ Die Wahrscheinlichkeit das ein Element auf die nächste Liste Kommt ist $1/2$, da die Elemente pro Liste halbiert werden.\\
$\Rightarrow$ Die Wahrscheinlichkeit, dass ein Element in Ebene $"i"$ erscheint ist dann:\\
$Pr= \frac{1}{2} * \frac{1}{2} * \frac{1}{2} * \frac{1}{2} * \frac{1}{2} * ... * \frac{1}{2} = (\frac{1}{2})^i$\\
$Pr = L_1 * L_2 * L_3 * L_4 * L_5 * ... * L_i$\\

\textbf{Gesucht:} Wieviele Knoten besitzt eine Skipliste mit $n$ Einträgen?\\

\textbf{Lösung:}
\begin{itemize}
	\item Ein Element auf der Ebene $0,1,2$ zähl als 3 Knoten $-$ ein Knoten pro Ebene
	\item Ein Elment ist auf Ebene $"i"$ mit der Wahrscheinlichkeit $(1/2)^i$
	\item Wenn wir den Erwartungswert für die gesamte Knotenanzahl haben wollen, addieren wir die Wahrscheinlichkeiten
	\item $\Rightarrow$ $1+ \frac{1}{2} + \frac{1}{4} + \frac{1}{8} + \frac{1}{16} + ... = \sum_{i=0}^{\infty}(1/2)^i$
	\item Das ist eine geometrische Reihe, die Konvergiert (einen Grenzwert bildet)
\end{itemize}

\textbf{geometrische Reihe:} $\sum_{i=0}^{\infty} q^k$\\
$\rightarrow$ $|q| < 1$ Konvergiert (nähert sich Grenzwert)\\
$\rightarrow$ $|q| > 1$ Divergiert (kein Grenzwert)\\
$\Rightarrow$ Grenzwert: $S= \frac{1}{1-q}$\\
$\rightarrow$ $S= \frac{1}{1-1/2} = \frac{1}{1/2} = 2$\\

Jedes Element erzeugt im Schnitt also 2 Knoten. Die gesamte Anzahl an Knoten in einer Skipliste ist dann Anzahl Elmente * Grenzwert für ein Element:\\
$n * 2 = 2n$ $\rightarrow$ für die O-Notation sind Konstanten irrelevant, also ist der Erwartungswert $O(n)$

\begin{flushright}
$\Box$
\end{flushright}


\newpage
\noindent
\textbf{b)} Zeigen Sie, dass für alle $j \geq 1$ die Wahrscheinlichkeit, dass $L$ aus mindestens $j$ Listen besteht, höchstens $\frac{n}{2_j-1}$ ist.\\
Hinweis: Für Ereignisse $[A_1,..., A_l]$ gilt: $Pr[A_{1} \cup ... \cup A_{l}] \geq \sum_{i=1}^{l}Pr[A_i]$\\
 
\begin{enumerate}
\item Wenn die Liste $L$ mindestens $j$ Ebenen/Verkettete Listen hat, dann muss mindestens ein Knoten $n_i$ aus $L$ aus der Basis-Liste $j-1$ mal in eine nächst höchere Ebene überführt worden sein.
   d.h bei einem Bernoulisexperiment (Münzwurf) wurde $j-1$ mal das selbe Ergebnis erziehlt und ein Knoten $n_i$ wurde $j-1$ mal in eine nächst höchere Ebene überführt.
\item Die Wahrscheinlichkeit $p$, dass ein Knoten $n_i$ in eine nächsthöchere Ebene überführt wird liegt bei 50\%. $\Rightarrow$ $p(Erfolg) = 1/2$
   Für den Erwartungswert, dass ein Knoten $n_i$ 1-mal in eine nächsthöhere Ebene überführt wird gilt: $E[x] = p(Erfolg) = 1/2$
   Nun wird betrachtet, dass ein Knoten $j-1$ mal überführt wird. Für diesen Erwartungswert gilt: $E[x] = j-1*p(Erfolg)$
   $E[x] = 1/2 * 1/2 * ... * 1/2 (j-1 mal)$
   $E[x] = 1/2^(j-1)$
\item Für die Anzahl der zu erwartenden Knoten, die $j-1$ mal überführt worden gilt:
   $X_i = 1$ wenn die $n_i$ $j-1$ mal hintereinander überführt worden ist. ansonsten $X_i = 0$
   Für die Summe $X$ aller $X_i$ (alle Knoten die $j-1$ mal überführt worden sind) gilt: $E[x] = n*1/2^(j-1) = n/2^(j-1)$
\item Für den Fall, dass eine Liste $L$ mit n Einträgen bei $j \geq 1$ Versuchen aus mindestens $j$ Listen besteht gilt: 
   $Pr(X \geq 1) \leq E[x] = n/2^(j-1)$
\end{enumerate}







