\section*{Problem 3: Implementierung von Tries} 


\noindent
Beschreiben Sie kurz, wie man konkret die Operationen put(s, v), get(s), remove(s)
und succ(s) auf unkomprimierten Tries implementieren kann. Dabei ist s jeweils
eine nichtleere Zeichenkette und v ein Wert aus einer endlichen Wertemenge V .
Geben Sie die Laufzeiten an.\\

\section*{Grundlegende Trie-Operationen}

Die Bearbeitung dieser Aufgabe orientiert sich an der Darstellung von Pseudocode aus dem Skript.

\subsection*{put / add}

\textbf{Ziel:} Fügt einen String mit einem Wert in einen Trie ein. Es wird angenommen, dass der String keine Leerzeichen enthält.

\begin{lstlisting}[language=Python]
put(s, v):
    current = Trie.root              # Beginne bei der Wurzel des Trie, diese ist immer leer.
    for char in s:                   # Iteriere ueber die einzelnen Buchstaben des Strings.
        if char not in current:     # Existiert kein Eintrag fuer diesen Buchstaben ...
            current[char] = {}      # ... dann erstelle einen neuen Knoten (Dictionary).
        current = current[char]     # Gehe in das naechste Level weiter.
    current["$"] = v                # Am Ende: Markiere das Ende des Strings mit "$" und speichere den Wert v.
\end{lstlisting}

\textbf{Laufzeit:} $O(|s|)$ — Die Schleife läuft einmal für jeden Buchstaben im String $s$ $\Rightarrow$ lineares Wachstum.

\subsection*{get / search}

\textbf{Ziel:} Überprüft, ob ein gegebener String im Trie enthalten ist. Gibt den zugehörigen Wert zurück, falls vorhanden.

\begin{lstlisting}[language=Python]
get(s):
    current = Trie.root
    for char in s:                   # Iteriere durch den Trie.
        if char not in current:
            return -1                # String ist nicht im Trie vorhanden.
        current = current[char]
    return current["$"]              # Gib den gespeicherten Wert zurueck.
\end{lstlisting}

\textbf{Laufzeit:} $O(|s|)$ — Linear zur Länge des eingegebenen Strings.

\subsection*{succ(s)}

\textbf{Ziel:} Findet den nächsten String im Trie (alphabetisch sortiert), der mit dem Präfix $s$ beginnt.

\begin{enumerate}
    \item Iteriere durch den Trie wie bei \texttt{get()} oder \texttt{put()}, bis der Präfix $s$ gefunden wurde.
    \item Falls $s$ ein vollständiger String ist:
    \begin{itemize}
        \item Suche von diesem Knoten aus den alphabetisch kleinsten Nachfolger (Knoten mit \texttt{"\$"}).
    \end{itemize}
    \item Falls kein solcher String existiert oder $s$ unvollständig ist:
    \begin{itemize}
        \item Gehe zurück zur letzten Verzweigung.
        \item Suche dort den nächsten möglichen (alphabetisch kleinsten) String.
    \end{itemize}
\end{enumerate}

\textbf{Laufzeit:} $O(|s| + |\text{result}|)$ — Linear zur Länge des Präfixes plus Länge des gesuchten Ergebnisses.

\subsection*{remove(s)}

\textbf{Ziel:} Entfernt einen gegebenen String aus dem Trie (falls vorhanden).

\begin{enumerate}
    \item Iteriere durch den Trie wie bei \texttt{get()}, um die Endmarkierung des Strings zu finden.
    \begin{itemize}
        \item Falls keine Endmarkierung gefunden wird, verlasse die Funktion.
    \end{itemize}
    \item Wird das Ende gefunden:
    \begin{itemize}
        \item Lösche das \texttt{"\$"}-Markierung.
        \item Gehe rekursiv von unten (Blatt) zur Wurzel:
        \begin{itemize}
            \item Wenn ein Knoten keine Kindknoten mehr hat $\Rightarrow$ lösche ihn.
            \item Sobald ein Knoten noch andere Kanten oder ein \texttt{"\$"} enthält $\Rightarrow$ stoppe.
        \end{itemize}
    \end{itemize}
\end{enumerate}

\textbf{Laufzeit:} $O(|s|)$ — Die Löschoperation ist ebenfalls linear, da Schleife und Rekursion nicht verschachtelt sind. $\Rightarrow$ O($2|s|$) = O($|s|$)
