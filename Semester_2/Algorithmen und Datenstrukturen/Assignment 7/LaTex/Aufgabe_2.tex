\definecolor{codegreen}{rgb}{0,0.6,0}
\definecolor{codegray}{rgb}{0.5,0.5,0.5}
\definecolor{codepurple}{rgb}{0.58,0,0.82}
\definecolor{backcolour}{rgb}{0.95,0.95,0.92}

\lstdefinestyle{mystyle}{
    backgroundcolor=\color{backcolour},   
    commentstyle=\color{codegreen},
    keywordstyle=\color{magenta},
    numberstyle=\tiny\color{codegray},
    stringstyle=\color{codepurple},
    basicstyle=\ttfamily\footnotesize,
    breakatwhitespace=false,         
    breaklines=true,                 
    captionpos=b,                    
    keepspaces=true,                 
    numbers=left,                    
    numbersep=5pt,                  
    showspaces=false,                
    showstringspaces=false,
    showtabs=false,                  
    tabsize=2
}
\lstset{style=mystyle}

\section*{{Problem 2: Implementierung einer Hashtabelle}} 


\noindent
\textbf{a)} Implementieren Sie eine Hashtabelle mit Verkettung in Scala. Benutzen Sie dazu die Funktion hashCode, die von allen Objekten in Scala zur Verfügung gestellt wird.\\
Gestalten Sie Ihre Implementierung so, dass sich die Größe der Hashtabelle wählen lässt, und implementieren Sie mit mindestens zwei verschiedenen Kompressionsfunktionen.\\

\noindent
\textbf{b)} Erweitern Sie Ihre Implementierung so, dass die Größe der Hashtabelle dynamisch angepasst wird, um einen Ladefaktor zwischen 1 und 3 zu garantieren (sobald mindestens 20 Einträge in der Hashtabelle vorhanden sind). Welche Strategie wählen Sie, um Ihre Hashtabelle anzupassen?\\



\noindent
\textbf{Datei: Hashtable.scala}
\lstinputlisting[language=Scala]{./Code/Hashtable.scala}