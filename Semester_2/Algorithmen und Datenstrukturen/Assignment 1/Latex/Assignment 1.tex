\documentclass[a4paper]{assignment}
\coursetitle{Algorithmen und Datenstrukturen}
\courselabel{AuD}
\exercisesheet{Assignment 1}{Algorithmen und Datenstrukturen}
\student{Moritz Ruge \& Lennard}
\school{Bachelor Informatik}
\university{Freie Universitäten Berlin}
\semester{Sommer Semester 2025}
\date{20 April 2025}
%\usepackage[pdftex]{graphicx}
%\usepackage{subfigure} 
\usepackage[backend=biber,style=numeric,url=true]{biblatex} % für Quellenangaben und Bibliotheken
\usepackage[]{hyperref} % für hyperlinks
\usepackage{array} % für die Tabelle
\usepackage{xcolor} %für Farben bei Text
\addbibresource{ref.bib}  % Add your .bib file here

% Paket für die Baumdiagramdarstellung
\usepackage{tikz}
\usetikzlibrary{trees}


\hypersetup{
    colorlinks=true,
    linkcolor=blue,
    filecolor=magenta,      
    urlcolor=cyan,
    pdftitle={Overleaf Example},
    pdfpagemode=FullScreen,
    }

\begin{document}

% Aufgabe 1 Binäre Suchbäume)
\begin{problemlist}

\pbitem{Binäre Suchbäume}

	\begin{enumerate}
			% Aufgabe 1 a
			\item Fügen Sie die Schlüssel A, L, G, O, D, T, S, X, Y, Z in dieser Reihenfolge in einen anfangs leeren binären Suchbaum ein. Löschen Sie sodann die Schlüssel Z, A, L. Zeichnen Sie den Baum nach jedem Einfüge- und Löschvorgang.

			% Antwort 1 a			
			\begin{answer}

\begin{itemize}
\item Einfügen des gesamten Schlüssels:
\end{itemize}

% Alle schlüssel eingefügt
\begin{tikzpicture}[scale=0.5]

\node {A}
	child[missing]
	child{ node{L}
	  child{ node{G}
	    child{ node{D}}
	    child[missing]}
	  child{ node{O}
	    child[missing]
	    child{ node{T}
	      child{node{S}}
	      child{node{X}
	        child[missing]
	        child{node{Y}
	          child[missing]
	          child{node{Z}}}}}}};
\end{tikzpicture}

% Löschen von Schlüssel Z
\begin{itemize}
\item Löschen von Schlüssel $Z$
\end{itemize}

\begin{tikzpicture}[scale=0.5]

\node {A}
	child[missing]
	child{ node{L}
	  child{ node{G}
	    child{ node{D}}
	    child[missing]}
	  child{ node{O}
	    child[missing]
	    child{ node{T}
	      child{node{S}}
	      child{node{X}
	        child[missing]
	        child{node{Y}}}}}};
\end{tikzpicture}

% Löschen von Schlüssel A
\begin{itemize}
\item Löschen von Schlüssel $A$
\end{itemize}

\begin{tikzpicture}[scale=0.5]

\node {L}
	  child{ node{G}
	    child{ node{D}}
	    child[missing]}
	  child{ node{O}
	    child[missing]
	    child{ node{T}
	      child{node{S}}
	      child{node{X}
	        child[missing]
	        child{node{Y}}}}};
\end{tikzpicture}

% Löschen von Schlüssel L
\begin{itemize}
\item Löschen von Schlüssel $L$
\end{itemize}

\begin{tikzpicture}[scale=0.5]

\node {G}
	    child{ node{D}
	    child[missing]}
	  child{ node{O}
	    child[missing]
	    child{ node{T}
	      child{node{S}}
	      child{node{X}
	        child[missing]
	        child{node{Y}}}}};
\end{tikzpicture}

\end{answer}	

			% Aufgabe 1 b
			\item Seien T1 und T2 zwei binäre Suchbäume, in denen jeweils die gleiche Menge S von Einträgen gespeichert ist, mit $|S| = n$. Zeigen Sie: Es gibt eine Folge von höchstens 2n einfachen Rotationen, die T1 nach T2 überführt. \\
	Zusatzfrage (5 Zusatzpunkte): Geht es auch mit weniger Rotationen?


			% Antwort 1 b
			\begin{answer}



			\end{answer}	
	\end{enumerate}
	
% Aufgabe 2 Geordnete Wörterbücher und Sortieren
\pbitem{Geordnete Wörterbücher und Sortieren}

\begin{enumerate}

\item Zeigen Sie: Wenn eine Implementierung des abstrakten Datentypen geordnetes Wörterbuch zur Verfügung steht, dann kann man diese verwenden, um eine gegebene Folge von n Elementen aus einer geordneten Menge zu sortieren.

% Antwort 2 a
\begin{answer}

\end{answer}

\item Aus “Konzepte der Programmierung” kennen Sie eine untere Schranke für vergleichsbasiertes Sortieren. Wie lautet diese?

% Antwort 2 b
\begin{answer}
$A$ sei ein vergleichbasierter Sortieralgorithmus und $n$ die Eingabe Größe. Dann Existiert eine Eingabe $I$ sodass $A$ mindestens $\frac{n}{2} (\log n - 1)$ vergleiche braucht. \\
Das zeigt, dass das Sortier Problem eine lower bound von $n \log n$ hat.
\end{answer}

\item Kombinieren Sie (a) und (b), um die untere Schranke für vergleichsbasiertes Sortieren auf geordnete Wörterbücher zu übertragen. Was folgt daraus über binäre Suchbäume?

% Antwort 2 c
\begin{answer}

\end{answer}

\end{enumerate}

\pbitem{Manipulation elementarer Funktionen}

Finden Sie Paare von äquivalenten Termen und formen Sie diese schrittweise ineinander um. Geben Sie die verwendeten Regeln an.

\[
\log_a\left(n^{\log_b a}\right), \quad b{\sqrt{a^n}}, \quad \frac{a^n}{a^m}
\]


	
	
	
\end{problemlist}

%Quellen Reference
%\newpage
%\printbibliography

\end{document}