\usepackage[left=3cm, right=2.5cm]{geometry} % Hier die Ränder definieren

%\usepackage[pdftex]{graphicx}
%\usepackage{subfigure} 
\usepackage[backend=biber,style=numeric,url=true]{biblatex} % für Quellenangaben und Bibliotheken
\usepackage[]{hyperref} % für hyperlinks
\usepackage{array} % für die Tabelle
\addbibresource{ref.bib}  % Add your .bib file here

% Paket für die Baumdiagramdarstellung
\usepackage{tikz}
\usetikzlibrary{trees}

% https://chatgpt.com/c/68191c8e-8760-8002-843b-e82091955567
\usepackage{booktabs} % für schönere Tabellenlinien
\usepackage{amsmath}  % für mathematische Symbole wie \left[ \right]
\usepackage{amssymb}

%%%% Packet um die Farbe von text zu ändern %%%%
%\textcolor{red}{text}
%\definecolor{mygreen}{RGB}{0,150,0}
%\textcolor{mygreen}{Dies ist grüner Text mit eigener Farbe.}
%{\color{blue}
%Dies ist ein ganzer Absatz in Blau.
%}
\usepackage{xcolor}

%%% draws layout frames
%\usepackage{showframe}



\hypersetup{
    colorlinks=true,
    linkcolor=blue,
    filecolor=magenta,      
    urlcolor=cyan,
    pdftitle={Overleaf Example},
    pdfpagemode=FullScreen,
    }