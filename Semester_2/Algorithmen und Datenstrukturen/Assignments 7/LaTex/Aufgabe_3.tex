\definecolor{codegreen}{rgb}{0,0.6,0}
\definecolor{codegray}{rgb}{0.5,0.5,0.5}
\definecolor{codepurple}{rgb}{0.58,0,0.82}
\definecolor{backcolour}{rgb}{0.95,0.95,0.92}

\lstdefinestyle{mystyle}{
    backgroundcolor=\color{backcolour},   
    commentstyle=\color{codegreen},
    keywordstyle=\color{magenta},
    numberstyle=\tiny\color{codegray},
    stringstyle=\color{codepurple},
    basicstyle=\ttfamily\footnotesize,
    breakatwhitespace=false,         
    breaklines=true,                 
    captionpos=b,                    
    keepspaces=true,                 
    numbers=left,                    
    numbersep=5pt,                  
    showspaces=false,                
    showstringspaces=false,
    showtabs=false,                  
    tabsize=2
}
\lstset{style=mystyle}

\section*{Problem 3: Lineares Sondieren und Löschen} 


\noindent
\textbf{a)} In der Vorlesung haben Sie eine Strategie gesehen, mit der das Löschen in einer Hashtabelle mit linearem Sondieren umgesetzt werden kann: Gelöschte Elemente werden durch einen eigenen Eintrag markiert und in den Einfüge- und Lookup Routinen speziell behandelt.\\
Beschreiben Sie eine alternative Methode, bei welcher der gelöschte Eintrag ggf. durch einen geeigneten anderen Eintrag ersetzt wird. Beschreiben sie Ihre Methode verbal und geben Sie Pseudocode. Geben Sie auch zwei interessante Beispiele, die zeigen, wie Ihre Methode funktioniert.\\

\noindent
\textbf{Lösung: Robin-Hood-Hashing mit Backwards-Shifting}\\

\textbf{Prinzip:} Es handelt sich hierbei um eine Hashtabelle mit \textit{Open Addressing}.\\
Beim Einfügen von neuen Schlüsseln wird für jeden Schlüssel der PSL (Probe Sequenz Lenght) berechnet um anzugeben, wie weit der Schlüssel von seiner eigentlichen Hashposition entfernt ist. Ziel ist es nun, das alle Schlüssel so nah wie möglich an ihrer "Home" Position bleiben. Beim Löschen von Schlüsseln wird geschaut ob es nachfolgend noch andere Schlüssel gibt, wenn nicht sind wir fertig, wenn doch wir auf deren PSL-Wert geschaut, ist dieser Größer als 0 Shiften wir den Schlüssel nach "Links" ( auf die Position des gelöschten Schlüssels ) und gegen in der Tabelle weiter, bis wir ein PSL-Wert von 0 oder einen Leeren Slot haben.\\

\noindent
\textbf{Einfügen:}

$\Rightarrow$ insert: $h(8)=1$
\begin{center}
\begin{tabular}{|c|c|c|c|c|c|c|}
\hline
Index & 0 & 1 & 2 & 3 & 4 & 5\\
\hline
T = & & $8_0$ & & & &\\
\hline
\end{tabular}
\end{center}

$\Rightarrow$ insert: $h(20)=1$
\begin{multicols}{2}
\begin{center}
\begin{tabular}{|c|c|c|c|c|c|c|}
\hline
Index & 0 & 1 & 2 & 3 & 4 & 5\\
\hline
T = & & $8_0$ & & & &\\
\hline
& & $20_0 \rightarrow$ & & & &\\
\hline
\end{tabular}
\end{center}


\columnbreak
\begin{center}
\begin{tabular}{|c|c|c|c|c|c|c|}
\hline
Index & 0 & 1 & 2 & 3 & 4 & 5\\
\hline
T = & & $8_0$ & $20_1$ & & &\\
\hline
\end{tabular}
\end{center}
$\Rightarrow$ Wir verschieben $20$ nach rechts und erhöhen den PSL um 1
\end{multicols}

$\Rightarrow$ insert: $h(3)=1$
\begin{multicols}{2}
\begin{center}
\begin{tabular}{|c|c|c|c|c|c|c|}
\hline
Index & 0 & 1 & 2 & 3 & 4 & 5\\
\hline
T = & & $8_0$ & $20_1$ & & &\\
\hline
& & $3_0 \rightarrow$ & $3_1 \rightarrow$& & &\\
\hline
\end{tabular}
\end{center}


\columnbreak
\begin{center}
\begin{tabular}{|c|c|c|c|c|c|c|}
\hline
Index & 0 & 1 & 2 & 3 & 4 & 5\\
\hline
T = & & $8_0$ & $20_1$ & $3_2$ & &\\
\hline
\end{tabular}
\end{center}
$\Rightarrow$ Wir verschieben $3$ 2x nach rechts und erhöhen den PSL um 2
\end{multicols}



$\Rightarrow$ insert: $h(9)=0$
\begin{center}
\begin{tabular}{|c|c|c|c|c|c|c|}
\hline
Index & 0 & 1 & 2 & 3 & 4 & 5\\
\hline
T = & $9_0$ & $8_0$ & $20_1$ & & &\\
\hline
\end{tabular}
\end{center}



\newpage
$\Rightarrow$ insert: $h(66)=0$
\begin{multicols}{2}
\begin{center}
\begin{tabular}{|c|c|c|c|c|c|c|}
\hline
Index & 0 & 1 & 2 & 3 & 4 & 5\\
\hline
T = & $9_0$ & $8_0$ & $20_1$ & & &\\
\hline
& $66_0 \rightarrow$ &  & & & &\\
\hline
\end{tabular}
\end{center}


\columnbreak
\begin{center}
\begin{tabular}{|c|c|c|c|c|c|c|}
\hline
Index & 0 & 1 & 2 & 3 & 4 & 5\\
\hline
T = & $9_0$ & $66_1$ & $20_1$ & $8_2$ & &\\
\hline
&  & $8_0 \rightarrow$ & $8_1 \rightarrow$ & & &\\
\hline
\end{tabular}
\end{center}
\end{multicols}
$\Rightarrow$ Wir verschieben $66$ nach rechts und erhöhen den PSL um 1 $\rightarrow 66_1$\\
$\rightarrow$ Da $66_1$ höher ist als $8_0$ tauschen wir die Werte und verschieben die 8 nach rechts\\
$\rightarrow$ Da $8_1 = 20_1$ ist schieben wir den Schlüssel weiter nach rechts und erhöhen den PSL\\


%%% This determines the width of the ruler to be used as column separator, it's set to 0 by default. In the example a column whose width is 1pt is printed.
\setlength{\columnseprule}{1pt}
\noindent
\textbf{Deletion:}\\

$\Rightarrow$ delete: $h(9)$
\begin{multicols}{2}
\begin{center}


\begin{tabular}{|c|c|c|c|c|c|c|}
\hline
Index & 0 & 1 & 2 & 3 & 4 & 5\\
\hline
T = & $9_0$ & $8_0$ & $20_1$ & & &\\
\hline
& $\uparrow$ &  & & & &\\
\hline
\end{tabular}\\
$\rightarrow$ Lookup $9$
\end{center}

\begin{center}
\begin{tabular}{|c|c|c|c|c|c|c|}
\hline
Index & 0 & 1 & 2 & 3 & 4 & 5\\
\hline
T = &  & $66_1$ & $20_1$ & $8_2$ & &\\
\hline
& $\uparrow$ &  &  & & &\\
\hline
\end{tabular}\\
$\rightarrow$ Lösche den Schlüssel
\end{center}

\begin{center}
\begin{tabular}{|c|c|c|c|c|c|c|}
\hline
Index & 0 & 1 & 2 & 3 & 4 & 5\\
\hline
T = &  & $\leftarrow 66_1$ & $20_1$ & $8_2$ & &\\
\hline
&  & $\uparrow$ &  & & &\\
\hline
\end{tabular}\\
$\Rightarrow$ Schaue auf den folgenden Slot:\\
$\rightarrow$ ist dieser leer sind wir fertig\\
$\rightarrow$ ist dort ein Schlüssel, schauen wir auf den PSL $\geq$ 0, Shiften wir den Eintrag nach links
\end{center}

\columnbreak
\begin{center}
\begin{tabular}{|c|c|c|c|c|c|c|}
\hline
Index & 0 & 1 & 2 & 3 & 4 & 5\\
\hline
T = & $66_0$ &  & $\leftarrow 20_1$ & $8_2$ & &\\
\hline
&  &  & $\uparrow$ & & &\\
\hline
\end{tabular}\\
$\rightarrow$ PSL $\geq$ 0, shift nach links\\
\end{center}


\begin{center}
\begin{tabular}{|c|c|c|c|c|c|c|}
\hline
Index & 0 & 1 & 2 & 3 & 4 & 5\\
\hline
T = & $66_0$ & $20_0$ &  & $\leftarrow 8_2$ & &\\
\hline
&  &  &  & $\uparrow$ & &\\
\hline
\end{tabular}\\
$\rightarrow$ PSL $\geq$ 0, shift nach links\\
\end{center}

\begin{center}
\begin{tabular}{|c|c|c|c|c|c|c|}
\hline
Index & 0 & 1 & 2 & 3 & 4 & 5\\
\hline
T = & $66_0$ & $20_0$ & $8_1$ &  & &\\
\hline
&  &  &  &  & $\uparrow$ &\\
\hline
\end{tabular}\\
$\rightarrow$ Slot ist leer, Fertig!\\
\end{center}
\end{multicols}



\noindent
\textbf{Pseudocode:}\\

\noindent
\textbf{put(k,v)}

\begin{lstlisting}[language=Scala]
def robin_hood_insert(key: K, value: V): Unit =
	var index = hash(key)
	var currentEntry = Entry(keyy, value, 0)
	
	while true do
		table(index) match
			case None =>
				table(index) = Some(currentEntry)
				return
			
			case Some(existingEntry) =>
				if existingEntry.key == currentEntry.key then
					table(index) = Some(currentEntry)
					return
				
				if currentEntry.probeLength > existingEntry.probeLength then
					// Robin Hood Swap
					table(index) = Some(currentEntry)
					currentEntry = existingEntry
					
				index = (index + 1) % size
				currentEntry = currentEntry.copy(probeLength = currentEntry.probeLength + 1)
\end{lstlisting}

\noindent
\textbf{delete(k)}

\begin{lstlisting}[language=Scala]
def delete(key: k): Unit =
	var index = hash(key)
	var probeLength = 0
	
	// als erstes muessen wir den Schluessel finden - loop durch die Tabelle
	while true do
		table(index) match
			case None => return // Schluessel nicht vorhanden - Fertig
			
			case Some(entry) => 
				if entry.key == key then // Wir haben den Schluessel gefunden - Delete
					removeAt(index)
					return // breche aus loop aus
				
				// Wenn die probeLength groesser ist als die des vorhandenen Elements
				// koennen wir abbrechen, da der Schluessel nicht mehr vorkommen kann
				if probeLength > entry.probeLength then return 
				
				index = (index + 1) % size
				probeLength += 1
				
				
private def removeAt(startIndex: Int): Unit =
	var i = startIndex
	var next = (i+1) % size
	
	// checken ob das naechste Element in der Tabelle NICHT leer ist UND ob die probelength
	// groesser als 0 ist -> Ansonsten brechen wir den Loop und sind Fertig
	while table(next).nonEmpty && table(next).get.grobeLength > 0 do
		val entry = table(next).get // holen den naechsten Eintrag aus der Tabelle
		tabel(i) = Some(entry.copy(probeLength = entry.probeLength -1 )) // wir kopieren den naechsten Eintrag auf die jetztige Stelle und verringern seinen PSL Wert
		i = next // wir setzen den i Wert auf die naechste Position
		next = (next + 1) % size // setzen die next Variable auf seine neue Position also + 1
	
	table(i) = None
\end{lstlisting}