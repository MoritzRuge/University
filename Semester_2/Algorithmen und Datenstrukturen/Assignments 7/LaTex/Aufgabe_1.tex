\section*{{Problem 1: Hashing im Selbstversuch II}}

\noindent
\textbf{a)} Fügen Sie nacheinander die Schlüssel 10, 22, 31, 4, 15, 28, 17, 88, 59 in eine Hashtabelle der Größe 11 ein. Die Hashfunktion sei $h(k) = k \mod 11$. Die Konflikte werden durch offene Adressierung mit linearem Sondieren gelöst.\\

\noindent
\textbf{b)} Fügen Sie nacheinander die Schlüssel 10, 22, 31, 4, 15, 29, 17, 88, 59 in eine Hashtabelle der Größe 11 ein. Die Konflikte werden durch Kuckuck gelöst, mit $h1(k) = k \mod 11$ und $h2(k) = (k \mod 13) \mod 11$.\\

\noindent
\textit{Illustrieren Sie jeweils die einzelnen Schritte.\\
	Hinweis: Pseudocode für die Hashoperation findet sich im Skript.}