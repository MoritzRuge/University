\section*{{Problem 1: Hashing im Selbstversuch II}}

\noindent
\textbf{a)} Fügen Sie nacheinander die Schlüssel 10, 22, 31, 4, 15, 28, 17, 88, 59 in eine Hashtabelle der Größe 11 ein. Die Hashfunktion sei $h(k) = k \mod 11$. Die Konflikte werden durch offene Adressierung mit linearem Sondieren gelöst.\\



\noindent
\textbf{Lösung:}\\ "  " : Empty, * : Deleted\\
\begin{center}
\begin{tabular}{|c|c|c|c|c|c|c|c|c|c|c|c|c|}
\hline
Index & 0 & 1 & 2 & 3 & 4 & 5 & 6 & 7 & 8 & 9 & 10\\
\hline
T = & & & & & & & & & & & \\
\hline
\end{tabular}
\end{center}

\begin{enumerate}
\item insert (10,v)

\begin{center}
\begin{tabular}{|c|c|c|c|c|c|c|c|c|c|c|c|c|}
\hline
Index & 0 & 1 & 2 & 3 & 4 & 5 & 6 & 7 & 8 & 9 & 10\\
\hline
T = & & & & & & & & & & & 10\\
\hline
\end{tabular}
\end{center}

\item insert (22,v)

\begin{center}
\begin{tabular}{|c|c|c|c|c|c|c|c|c|c|c|c|c|}
\hline
Index & 0 & 1 & 2 & 3 & 4 & 5 & 6 & 7 & 8 & 9 & 10\\
\hline
T = & 22 & & & & & & & & & & 10\\
\hline
\end{tabular}
\end{center}

\item insert (31,v)

\begin{center}
\begin{tabular}{|c|c|c|c|c|c|c|c|c|c|c|c|c|}
\hline
Index & 0 & 1 & 2 & 3 & 4 & 5 & 6 & 7 & 8 & 9 & 10\\
\hline
T = & 22 & & & & & & & & & 31 & 10\\
\hline
\end{tabular}
\end{center}

\item insert (4,v)

\begin{center}
\begin{tabular}{|c|c|c|c|c|c|c|c|c|c|c|c|c|}
\hline
Index & 0 & 1 & 2 & 3 & 4 & 5 & 6 & 7 & 8 & 9 & 10\\
\hline
T = & 22 & & & & 4 & & & & & 31 & 10\\
\hline
\end{tabular}
\end{center}

\item insert (15,v)

\begin{center}
\begin{tabular}{|c|c|c|c|c|c|c|c|c|c|c|c|c|}
\hline
Index & 0 & 1 & 2 & 3 & 4 & 5 & 6 & 7 & 8 & 9 & 10\\
\hline
T = & 22 & & & & 4 & 15 & & & & 31 & 10\\
\hline
\end{tabular}
\end{center}
$\Rightarrow$ index 4: 4 $\neq$ 15 $\Rightarrow$ Index 5: Empty $\rightarrow$ Index 5 = 15

\item insert (28,v)

\begin{center}
\begin{tabular}{|c|c|c|c|c|c|c|c|c|c|c|c|c|}
\hline
Index & 0 & 1 & 2 & 3 & 4 & 5 & 6 & 7 & 8 & 9 & 10\\
\hline
T = & 22 & & & & 4 & 15 & 28 & & & 31 & 10\\
\hline
\end{tabular}
\end{center}

\item insert (17,v)

\begin{center}
\begin{tabular}{|c|c|c|c|c|c|c|c|c|c|c|c|c|}
\hline
Index & 0 & 1 & 2 & 3 & 4 & 5 & 6 & 7 & 8 & 9 & 10\\
\hline
T = & 22 & & & & 4 & 15 & 28 & 17 & & 31 & 10\\
\hline
\end{tabular}
\end{center}
$\Rightarrow$ Index 6: 28 $\neq$ 17 $\rightarrow$ Index 7: Empty $\rightarrow$ Index 7 = 17

\item insert (88,v)

\begin{center}
\begin{tabular}{|c|c|c|c|c|c|c|c|c|c|c|c|c|}
\hline
Index & 0 & 1 & 2 & 3 & 4 & 5 & 6 & 7 & 8 & 9 & 10\\
\hline
T = & 22 & 88 & & & 4 & 15 & 28 & 17 & & 31 & 10\\
\hline
\end{tabular}
\end{center}
$\Rightarrow$ Index 0: 22 $\neq$ 88 $\rightarrow$ Index 1: Empty $\rightarrow$ Index 1 = 88

\item insert (59,v)

\begin{center}
\begin{tabular}{|c|c|c|c|c|c|c|c|c|c|c|c|c|}
\hline
Index & 0 & 1 & 2 & 3 & 4 & 5 & 6 & 7 & 8 & 9 & 10\\
\hline
T = & 22 & 88 & & & 4 & 15 & 28 & 17 & 59 & 31 & 10\\
\hline
\end{tabular}
\end{center}
$\Rightarrow$ Index 4: 4 $\neq$ 59 $\rightarrow$ Index 8: Empty $\rightarrow$ Index 8 = 59


\end{enumerate}

\noindent
\textbf{b)} Fügen Sie nacheinander die Schlüssel 10, 22, 31, 4, 15, 29, 17, 88, 59 in eine Hashtabelle der Größe 11 ein. Die Konflikte werden durch Kuckuck gelöst, mit $h1(k) = k \mod 11$ und $h2(k) = (k \mod 13) \mod 11$.\\

\noindent
\textit{Illustrieren Sie jeweils die einzelnen Schritte.\\
	Hinweis: Pseudocode für die Hashoperation findet sich im Skript.}\\
	
\noindent
\textbf{Lösung:}\\ "  " : Empty, * : Deleted\\
\begin{center}
\begin{tabular}{|c|c|c|c|c|c|c|c|c|c|c|c|c|}
\hline
Index & 0 & 1 & 2 & 3 & 4 & 5 & 6 & 7 & 8 & 9 & 10\\
\hline
T = & & & & & & & & & & & \\
\hline
\end{tabular}
\end{center}

\begin{enumerate}

\item insert (10)

$\Rightarrow h_1(k) = k \mod 11 = 10 \mod 11 = 10$ 
\begin{center}
\begin{tabular}{|c|c|c|c|c|c|c|c|c|c|c|c|c|}
\hline
Index & 0 & 1 & 2 & 3 & 4 & 5 & 6 & 7 & 8 & 9 & 10\\
\hline
T = & & & & & & & & & & & 10\\
\hline
\end{tabular}
\end{center}

\item insert (22)

$\Rightarrow h_1(k) = k \mod 11 = 22 \mod 11 = 0$ 
\begin{center}
\begin{tabular}{|c|c|c|c|c|c|c|c|c|c|c|c|c|}
\hline
Index & 0 & 1 & 2 & 3 & 4 & 5 & 6 & 7 & 8 & 9 & 10\\
\hline
T = & 22 & & & & & & & & & & 10\\
\hline
\end{tabular}
\end{center}

\item insert (31)

$\Rightarrow h_1(k) = k \mod 11 = 31 \mod 11 = 9$ 
\begin{center}
\begin{tabular}{|c|c|c|c|c|c|c|c|c|c|c|c|c|}
\hline
Index & 0 & 1 & 2 & 3 & 4 & 5 & 6 & 7 & 8 & 9 & 10\\
\hline
T = & 22 & & & & & & & & & 31 & 10\\
\hline
\end{tabular}
\end{center}

\item insert (4)

$\Rightarrow h_1(k) = k \mod 11 = 4 \mod 11 = 4$ 
\begin{center}
\begin{tabular}{|c|c|c|c|c|c|c|c|c|c|c|c|c|}
\hline
Index & 0 & 1 & 2 & 3 & 4 & 5 & 6 & 7 & 8 & 9 & 10\\
\hline
T = & 22 &  &  &  & 4 &  &  &  &  & 31 & 10\\
\hline
\end{tabular}
\end{center}

\item insert (15)

$\Rightarrow h_1(15) = 15 \mod 11 = 4$ Platziere in Index $4$, $k'=4$.\\
$\rightarrow h_1(4') = 4 \mod 11 = 4$. Anwendung $h_2(k')$\\ 
$\rightarrow h_2(4') = (4 \mod 13) \mod 11 = 4$ Platz in Index $4$, $k' = 15$.\\
$\Rightarrow h_1(15') = 15 \mod 11 = 4$. Anwendung $h_2(k')$\\ 
$\rightarrow h_2(15') = (15 \mod 13) \mod 11 = 2$ Platz in Index $2$.\\
\begin{center}
\begin{tabular}{|c|c|c|c|c|c|c|c|c|c|c|c|c|}
\hline
Index & 0 & 1 & 2 & 3 & 4 & 5 & 6 & 7 & 8 & 9 & 10\\
\hline
T = & 22 &  & 15 &  & 4 &  &  &  &  & 31 & 10\\
\hline
\end{tabular}
\end{center}

\item insert (29)

$\Rightarrow h_1(k) = 29 \mod 11 = 7$ Platziere in Index $7$.
\begin{center}
\begin{tabular}{|c|c|c|c|c|c|c|c|c|c|c|c|c|}
\hline
Index & 0 & 1 & 2 & 3 & 4 & 5 & 6 & 7 & 8 & 9 & 10\\
\hline
T = & 22 &  & 15 &  & 4 &  &  & 29 &  & 31 & 10\\
\hline
\end{tabular}
\end{center}

\item insert (17)

$\Rightarrow h_1(k) = 17 \mod 11 = 6$ Platziere in Index $6$.
\begin{center}
\begin{tabular}{|c|c|c|c|c|c|c|c|c|c|c|c|c|}
\hline
Index & 0 & 1 & 2 & 3 & 4 & 5 & 6 & 7 & 8 & 9 & 10\\
\hline
T = & 22 &  & 15 &  & 4 &  & 17 & 29 &  & 31 & 10\\
\hline
\end{tabular}
\end{center}

\item insert (88)

$\Rightarrow h_1(88) = 88 \mod 11 = 0$ Platziere in Index $0$, $k' = 22$.\\
$\rightarrow h_1(22') = 22 \mod 11 = 0$. Anwendung $h_2(k')$\\ 
$\rightarrow h_2(22') = (22 \mod 13) \mod 11 = 9$ Platz in Index $9$, $k' = 31$.\\
$\Rightarrow h_1(31') = 31 \mod 11 = 9$ Anwendung $h_2(k')$\\
$\rightarrow h_2(31') = (31 \mod 13) \mod 11 = 5$ Platz in Index $5 \rightarrow$ Empty.\\


\begin{center}
\begin{tabular}{|c|c|c|c|c|c|c|c|c|c|c|c|c|}
\hline
Index & 0 & 1 & 2 & 3 & 4 & 5 & 6 & 7 & 8 & 9 & 10\\
\hline
T = & 88 &  & 15 &  & 4 & 31 & 17 & 29 &  & 22 & 10\\
\hline
\end{tabular}
\end{center}

\item insert (59)

$\Rightarrow h_1(59) = 59 \mod 11 = 4$ Platziere in Index $4$, $k' = 4$.\\
$\rightarrow h_1(4') = 4 \mod 11 = 4$. Anwendung $h_2(k')$\\ 
$\rightarrow h_2(4') = (4 \mod 13) \mod 11 = 4$ Platz in Index $4$, $k' = 59$.\\
$\Rightarrow h_1(59') = 59 \mod 11 = 4$ Anwendung $h_2(k')$.\\
$\rightarrow h_2(59') = (59 \mod 13) \mod 11 = 7$ Platz in Index $7$, $k' = 29$.\\
$\Rightarrow h_1(29') = 29 \mod 11 = 7$ Anwendung $h_2(k')$.\\
$\rightarrow h_2(29') = (29 \mod 13) \mod 11 = 3$ Platz in Index $3 \rightarrow$ Empty.\\

\begin{center}
\begin{tabular}{|c|c|c|c|c|c|c|c|c|c|c|c|c|}
\hline
Index & 0 & 1 & 2 & 3 & 4 & 5 & 6 & 7 & 8 & 9 & 10\\
\hline
T = & 88 &  & 15 & 29 & 4 & 31 & 17 & 59 &  & 22 & 10\\
\hline
\end{tabular}
\end{center}

\end{enumerate}