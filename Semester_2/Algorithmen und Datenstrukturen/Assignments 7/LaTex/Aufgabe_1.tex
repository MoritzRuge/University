\section*{{Problem 1: Hashing im Selbstversuch II}}

\noindent
\textbf{a)} Fügen Sie nacheinander die Schlüssel 10, 22, 31, 4, 15, 28, 17, 88, 59 in eine Hashtabelle der Größe 11 ein. Die Hashfunktion sei $h(k) = k \mod 11$. Die Konflikte werden durch offene Adressierung mit linearem Sondieren gelöst.\\



\noindent
\textbf{Lösung:}\\ "  " : Empty, * : Deleted\\
\begin{center}
\begin{tabular}{|c|c|c|c|c|c|c|c|c|c|c|c|c|}
\hline
Index & 0 & 1 & 2 & 3 & 4 & 5 & 6 & 7 & 8 & 9 & 10\\
\hline
T = & & & & & & & & & & & \\
\hline
\end{tabular}
\end{center}

\begin{enumerate}
\item insert (10,v)

\begin{center}
\begin{tabular}{|c|c|c|c|c|c|c|c|c|c|c|c|c|}
\hline
Index & 0 & 1 & 2 & 3 & 4 & 5 & 6 & 7 & 8 & 9 & 10\\
\hline
T = & & & & & & & & & & & 10\\
\hline
\end{tabular}
\end{center}

\item insert (22,v)

\begin{center}
\begin{tabular}{|c|c|c|c|c|c|c|c|c|c|c|c|c|}
\hline
Index & 0 & 1 & 2 & 3 & 4 & 5 & 6 & 7 & 8 & 9 & 10\\
\hline
T = & 22 & & & & & & & & & & 10\\
\hline
\end{tabular}
\end{center}

\item insert (31,v)

\begin{center}
\begin{tabular}{|c|c|c|c|c|c|c|c|c|c|c|c|c|}
\hline
Index & 0 & 1 & 2 & 3 & 4 & 5 & 6 & 7 & 8 & 9 & 10\\
\hline
T = & 22 & & & & & & & & & 31 & 10\\
\hline
\end{tabular}
\end{center}

\item insert (4,v)

\begin{center}
\begin{tabular}{|c|c|c|c|c|c|c|c|c|c|c|c|c|}
\hline
Index & 0 & 1 & 2 & 3 & 4 & 5 & 6 & 7 & 8 & 9 & 10\\
\hline
T = & 22 & & & & 4 & & & & & 31 & 10\\
\hline
\end{tabular}
\end{center}

\item insert (15,v)

\begin{center}
\begin{tabular}{|c|c|c|c|c|c|c|c|c|c|c|c|c|}
\hline
Index & 0 & 1 & 2 & 3 & 4 & 5 & 6 & 7 & 8 & 9 & 10\\
\hline
T = & 22 & & & & 4 & 15 & & & & 31 & 10\\
\hline
\end{tabular}
\end{center}
$\Rightarrow$ index 4: 4 $\neq$ 15 $\Rightarrow$ Index 5: Empty $\rightarrow$ Index 5 = 15

\item insert (28,v)

\begin{center}
\begin{tabular}{|c|c|c|c|c|c|c|c|c|c|c|c|c|}
\hline
Index & 0 & 1 & 2 & 3 & 4 & 5 & 6 & 7 & 8 & 9 & 10\\
\hline
T = & 22 & & & & 4 & 15 & 28 & & & 31 & 10\\
\hline
\end{tabular}
\end{center}

\item insert (17,v)

\begin{center}
\begin{tabular}{|c|c|c|c|c|c|c|c|c|c|c|c|c|}
\hline
Index & 0 & 1 & 2 & 3 & 4 & 5 & 6 & 7 & 8 & 9 & 10\\
\hline
T = & 22 & & & & 4 & 15 & 28 & 17 & & 31 & 10\\
\hline
\end{tabular}
\end{center}
$\Rightarrow$ Index 6: 28 $\neq$ 17 $\rightarrow$ Index 7: Empty $\rightarrow$ Index 7 = 17

\item insert (88,v)

\begin{center}
\begin{tabular}{|c|c|c|c|c|c|c|c|c|c|c|c|c|}
\hline
Index & 0 & 1 & 2 & 3 & 4 & 5 & 6 & 7 & 8 & 9 & 10\\
\hline
T = & 22 & 88 & & & 4 & 15 & 28 & 17 & & 31 & 10\\
\hline
\end{tabular}
\end{center}
$\Rightarrow$ Index 0: 22 $\neq$ 88 $\rightarrow$ Index 1: Empty $\rightarrow$ Index 1 = 88

\item insert (59,v)

\begin{center}
\begin{tabular}{|c|c|c|c|c|c|c|c|c|c|c|c|c|}
\hline
Index & 0 & 1 & 2 & 3 & 4 & 5 & 6 & 7 & 8 & 9 & 10\\
\hline
T = & 22 & 88 & & & 4 & 15 & 28 & 17 & 59 & 31 & 10\\
\hline
\end{tabular}
\end{center}
$\Rightarrow$ Index 4: 4 $\neq$ 59 $\rightarrow$ Index 8: Empty $\rightarrow$ Index 8 = 59


\end{enumerate}

\noindent
\textbf{b)} Fügen Sie nacheinander die Schlüssel 10, 22, 31, 4, 15, 29, 17, 88, 59 in eine Hashtabelle der Größe 11 ein. Die Konflikte werden durch Kuckuck gelöst, mit $h1(k) = k \mod 11$ und $h2(k) = (k \mod 13) \mod 11$.\\

\noindent
\textit{Illustrieren Sie jeweils die einzelnen Schritte.\\
	Hinweis: Pseudocode für die Hashoperation findet sich im Skript.}