%%%%% TODO %%%%%
% subsection a % Zwei Seiten layout ? damit die ganzen Bäume nicht so viel Platz wegnehmen

\section*{Problem 1: Rot-Schwarz Bäume und (2,4)-Bäume}

In Aufgabe 3 auf dem 3. Aufgabenblatt wurden rot-schwarz Bäume definiert.\\


\textbf{a)} Zeigen Sie: Rot-schwarz Bäume und (2, 4)-Bäume sind äquivalent. Genauer: es gibt eine lokale Transformation, welche Gruppen von Knoten im rot-schwarz Baum in Knoten im (2, 4)-Baum überführt, und umgekehrt. Geben Sie eine
solche Transformation an, und begründen Sie, dass Ihre Transformation die Bedingungen an rot-schwarz Bäume und an (2, 4)-Bäume erfüllt.\\

\noindent
In einem (2,4)-Baum gilt:\\
min: 2 Kindknoten und max: 4 Kindknoten\\
min Einträge: 1 und max Einträge: 3\\
Bei der Convertierung eines (2,4)-Baums in einen Rot-Schwarz-Baum müssen betrachtet werden:\\
k: Elternknoten; w: Kinderknoten\\
Convertierung:\\

\noindent
1-Knoten $\rightarrow$ rot-schwarz: Bei einem Knoten mit nur 1. Eintrag, wird der Knoten zu einem Schwarzknoten. $\rightarrow$\\
Ein 1-Knoten hat entweder 0 oder 2 Kindknoten\\

\begin{addmargin}[1em]{1em}
$-$ 0 Kindknoten: Schwarzer Blattknoten\\
$-$ 2 Kindknoten: für den 1-Knoten gilt: $v = (w_1,k_1,w_2)$, wobei $w_1 < k_1 < w_2 -> w_1$ wird zum linken Kindknoten von $k_1$ und $w_2$ wird zum rechten Kind von $k_1$ Die Farbe der Kindknoten wird vom Elternknoten bestimmt.\\
\end{addmargin}
\noindent
2-Knoten $\rightarrow$ red-black: Bei einem Knoten mit 2 Einträgen, wird 1 Knoten zum Elternknoten und der andere Knoten wird zum Kindknoten\\

\begin{addmargin}[1em]{1em}
$-$ Dabei gilt: $v = (k_1, k_2)$, wobei $k_1 < k_2$ $\rightarrow$ Daraus entstehen zwei Möglichkeiten:\\
(i) $k_1$: Elternknoten und $k_2$: rechter Kindknoten\\
(ii) $k_2$: Elternknoten und $k_1$: linker Kindknoten\\
$-$ Ein 2-Knoten hat 3 oder 0 Kindknoten. Für die Verteilung der Kindknoten gilt:\\
(i) Wenn $v$ keine Kindknoten hat, dann ist $v$ ein Blattknoten. Somit gilt für die Kinder von $v$, dass sie zu schwarzen Blattknoten werden, nachdem sie in den Rot-Schwarz-Baum überführt wurden.\\
(i) $v = (w_1,k_1,w_2,k_2,w_3) $\\
\end{addmargin}

\noindent
Zur Veranschaulichung:
\begin{verbatim}
```  
     |A|B|
    C  D  E     
```
\end{verbatim}

\begin{addmargin}[1em]{1em}
$-$ Für $v$ gibt es wieder zwei Möglichkeiten:\\
$-$ es gilt $w_1 < ... < k_2 < w_3$ $\rightarrow$ Größenordnung von links nach rechts\\
(1.) $k_1$ wird zum Elternknoten:\\
$w_1$: linker Kindknoten von $k_1$\\
$k_2$: rechter Kindknoten von $k_1$\\
$w_2$: linker Kindknoten von $k_2$\\
$w_3$: rechter Kindknoten von $k_2$\\
(2.) $k_2$ wird zum Elternknoten:\\
$k_1$: linker Kindknoten von $k_2$\\
$w_3$: rechter Kindknoten von $k_2$\\
$w_1$: linker Kindknoten von $k_1$\\
$w_2$: rechter Kindknoten von $k_1$\\
$-$ Die Farbe der Kindknoten wird vom Elternknoten bestimmt.\\
\end{addmargin}

\begin{addmargin}[1em]{1em}
3-Knoten $->$ red-black: Bei einem Knoten mit 2 Einträgen gilt: $v = (k_1,k_2,k_3)$, dabei wird $k_2$ zum Schwarzen Elternknoten, $k_1$ zum Roten linken Kindknoten von $k_2$ und $k_3$ zum rechten Kindknoten von $k_2$.\\
$-$ Ein 3-Knoten hat 4 oder 0 Kindknoten. Für diese gilt:\\
$-$ $w_1 < k_1 < w_2 < k_2 < w_3 < k_3 < w_4 $\\
$-$ Darus folgt:\\
\end{addmargin}
\begin{addmargin}[2em]{1em}
$w_1$: linker Kindknoten von $k_1$\\
$w_2$: rechter Kindknoten von $k_1$\\
$w_3$: linker Kindknoten von $k_3$\\
$w_4$: rechter Kindknoten von $k_3$\\
\end{addmargin}
\begin{addmargin}[1em]{1em}
$-$ Wenn v keine Kindknoten hat, dann ist v ein Blattknoten. Somit gilt für die Kinder von v, dass sie zu schwarzen Blattknoten werden, nachdem sie in den Rot--Schwarz-Baum überführt wurden.\\
$-$ Die Farbe der Kindknoten wird vom Elternknoten bestimmt.\\
\end{addmargin}

\noindent
Wurzel:\\
$-$ Die Wurzel verhält sich je nach Baum, genau wie ein Innerer Knoten oder ein Blatt Knoten und ist immer Schwarz. Eine spezielle Betrachtung ist daher nicht nötig. Die Farbwahl aller Knoten in einem in Rot-Schwarz überführten (2,4)-Baum geht von der Wurzel aus.\\

\noindent
Unter Anwendung dieser Regeln, lässt sich jeder (2,4) Baum in einen Rot-Schwarz-Baum überführen.\\
Beispiel: (2,4)-Baum aus der Aufgabe 2b -> Rot-Schwarz-Baum\\
Ich stelle Knoten im Rot-Schwarz-Baum als Tuppel (key,colour)\\

\begin{verbatim}
```
                 ---------------
                 | G    |    T |
                 ---------------
                /       |        \
               /        |         \
         | A | D | | L | O | S | | X | Y | Z |
```
```1. Wurzel -> Rot-Schwarz-Knoten Wähle G als Wurzel im Rot-Schwarz-Baum
                   (G,B)
                /        \
               /          \
         | A | D |       (T,R)
                       /       \
              | L | O | S | | X | Y | Z |
```2. Linkes Kind von G -> Rot-Schwarz-Knoten Wähle D als Elternknoten von A.
                   (G,B)
                /        \
               /          \
            (D,R)       (T,R)
          /           /       \
        (A,B)   | L | O | S | | X | Y | Z |
```3. Linkes und Rechtes Kind von R -> Rot-Schwarz-Knoten Wähle O als Elternknoten von L,S und Y als Elternknoten von X,Z
                   (G,B)
                /        \
               /          \
            (D,R)       (T,R)
          /           /       \
        (A,B)      (O,B)     (Y,B)
                  /     \   /     \
                |L|    |S| |X|    |Z|
```
```4. Wandle verbliebende (2,4)-Blätter in Rot-Schwarz-Blätter um. 
                   (G,B)
                /        \
               /          \
            (D,R)       (T,R)
          /           /       \
         /           /         \
       (A,B)      (O,B)       (Y,B)
       /   \     /     \     /     \
      B     B  (L,R) (S,R) (X,R)   (Z,R)
               /   \ /   \ /   \   /   \
              B    B B   B B   B  B     B
```
\end{verbatim}