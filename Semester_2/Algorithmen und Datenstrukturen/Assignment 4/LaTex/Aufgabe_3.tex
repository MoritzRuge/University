\section*{Problem 3: (a,b)-Bäume} 

\textbf{a)} Beschreiben Sie, wie man in einem (a, b)-Baum mit n Schlüsseln die Operation succ(k) implementieren kann. Was ist die Laufzeit?\\

$\Rightarrow$ succ(k) - finde den Nachfolge vom Schlüssel $k$

\begin{enumerate}
\item Suche den Knoten $"i"$, der den Eintrag $k$ enthält
\item Wenn $k$ nicht das größte Element in $u$ ist, schaue ob es noch ein Teilbaum zwischen dem Element $k$ und seinem direkten Nachfolger gibt.
	\begin{itemize}
		\item Wenn Nein $\rightarrow$ dann gib den direkten Nachfolger von $k$ zurück
		\item Wenn ja, gehe in den Teilbaum und gebe das kleinste Element zurück
	\end{itemize}
\item Wenn $k$ das größte Element in $u$ ist:
	\begin{itemize}
		\item wenn $i$ ein rechten Teilbaum besitzt, gehe in den rechten Teilbaum und gebe das kleinste Element zurück
		\item ansonsten gehe zum Elternknoten und suche das erste Element, das größer als $k$ ist und gebe es zurück
	\end{itemize}
\end{enumerate}

$\Rightarrow$ Die Laufzeit beträgt $O(logn)$, da die Höhe eines (a,b)-Baums $O(logn)$ ist. \\



\noindent
\textbf{b)} Beschreiben Sie, wie man in einem (a, b)-Baum mit n Schlüsseln die Operation $findRange(k1 , k2)$ implementieren kann, die alle Schlüssel $k$ liefert, für die
$k1 \leq k \leq k2$ ist. Die Laufzeit soll $O(b loga n + s)$ betragen. Dabei ist $s$ die Anzahl der gelieferten Schlüssel.\\

\begin{verbatim}
findRange(k_1,k_2):
 Sei T ein (a,b)-Baum mit Knoten v.
 Jeder Knoten besteht aus einem oder mehreren keys k_i, dabei sind die keys innerhalb 
 eines Knoten von links(kleinster) nach rechts(größter) key sortiert.
 Jeder key hat zusätzlich einen Verweis (ab hier pointer genannt) auf seine 
 Kinder- und Elternknoten.
 
 
 Pseusocode findRange(k_1,k_2):
 - Wenn gilt: k_1 == k_2 -> wenn die gesuchte Reichweite nur einen key beinhaltet 
 gibt die Position des keys mittels get(k_2) zurück, wenn sie existiert.
	return get(k_2)
	
 - erstelle eine leere Liste: found_keys = []
 - erstelle einen counter für keys: left_boarder = k_1
 
 - Finde den key k_i >= k_1 mit der Funktion get(left_boarder), um die Linke 
 Grenze zu bestimmen. get(left_boarder) ist notwendig, falls k_1 
 nicht Teil des (a,b)-Baumes ist.
   - solange kein k_i gefunden wurde wiederhole diesen Funktionsaufruf, bis 
   ein k_i gefunden wurde. Inkrementiere Dabei vor jeder nächsten Iteration 
   den counter left_boarder.
     - Wenn gilt: k_i == k_2 -> dann liegt die gesuchte Reichweite nicht im Baum
	return err
	
     - Wenn gilt: k_i >= k_1 -> hänge den key der Liste an und verlasse den Loop.
	found_keys.append(k_i)
	break
	
 - // Diese Suche hat im worst-case eine Zeitkomplexität von O(b*log_a(n)). O(get) = O(log(n)) 
 und dies muss maximal b-mal wiederholt werden, da der startkey k_i minimal 
 der kleinste key innerhalb eines Knotend v sein kann. Spätestens nach der b-ten
 Inkrementierung ist der key k_i == Elternkey des Knotens v, in dem sich k_i zu Beginn
  befindet und dieser ist größer als k_1.
 - Nun kann man iterative linear alle keys, für die gilt: k_j = found_nodes[0], 
 k_j <= k_2, an die Liste der gefundenen keys anhängen.
   - für Knoten v gilt: v = (k_1,...,k_l)
   - solange k_j <= k_2:
     - solange k_j <= k_l -> hänge k_j and die Liste der gefundenen keys an.
	found_keys.append(k_j) 
	
     - k_j = Elternkey von v
 - return found_keys // gibt die Liste aller keys zurück, die gefunden wurden 
 found_keys[0] = k_1 und found_keys[len(found_keys)-1] = k_2
 - Die Zeitkomplexität dieser Suche ist O(s*log(s)), wobei s die Anzahl der 
 besuchten keys ist. s-mal, da insgesamt s keys betrachtet wurden und 
 log(s)-mal, da innerhalb eines bestimmten Knoten nur ein bestimmter Teil von s 
 betrachtet wird.
   Die Dominate Klasse ist linear O(s)
 - Für die Gesamt Zeitkomplexität gilt: O(b*log_a(n) + s), weil die beiden Teile 
 der Funktion squentiell ablaufen und nicht ineinander verschachtelt sind. 
\end{verbatim}

	

\newpage
\noindent
\textbf{c)} Seien $T1$ und $T2$ zwei (a, b)-Bäume, und sei $S1$ die Schlüsselmenge von $T1$ und $S2$ die Schlüsselmenge von $T2$ . Sei $x$ ein weiterer Schlüssel. Alle Schlüssel in $S1$ sind kleiner als $x$, und alle Schlüssel in $S2$ sind größer als $x$. Beschreiben Sie eine Operation join, die aus $T1$, $T2$ und $x$ einen (a, b)-Baum für die Schlüsselmenge $S1 \cup \{x\} \cup S2$ erzeugt. Die Laufzeit sollte $O(b \log_{a} max\{|S1|, |S2|\})$ betragen. Hinweis: Betrachten Sie zunächst den Fall, dass $T1$ und $T2$ die gleiche Höhe haben. Achten Sie darauf, dass hinterher die (a, b)-Baum Eigenschaften wieder hergestellt werden.\\

\begin{enumerate}
\item Fall - $T1$ \& $T2$ haben die gleiche Höhe
	\begin{itemize}
	\item $\Rightarrow$ Sei $S1 < x < S2$
	\item Setzte den Schlüssel $x$ als Wurzel
	\item Das linke Kind wird der Baum $T1$ und das rechte Kind wird der Baum $T2$ $\Rightarrow$ $join(T1, x, T2)$
	\item Da die Bäume schon in (a,b)-Baum Ordnung sind, müssen wir hier auch nicht weiter überprüfen 
	\end{itemize}


\item Fall - $T1$ \& $T2$ haben unterschiedliche Höhe\\
Angenommen $T1$ ist höher als $T2$
\begin{itemize}
	\item durchlaufe den größeren Baum($T1$) bis zur Höhe von $T2$
	\item führe nun auf gleicher Höhe die $join$ Operation durch von $T1$, bis zur Höhe $h$, und $T2$ 
	\item die Restlichen Knoten von T1 werden danach eingefügt, dabei müssen wir die (a,b)-Baum Ordnung überprüfen, da Knoten jetzt zu viele Elemente haben können $\Rightarrow$ wir müssen nach oben hin mit Splits oder Merges arbeiten
\end{itemize}

\item[$-$] Laufzeitanalyse
\begin{itemize}
	\item Das durchsteigen eines (a,b,)-Baums zur Höhe $h$ hin dauert $O(\log_{a} max\{|S1|, |S2|\})$
	\item Das spliten oder Mergen auf dem Pfad nach oben braucht $O(b)$, da ein Knoten maximal $b$ Kinder haben kann.
	\item Gesamtlaufzeit: $O(b \log_{a} max\{|S1|, |S2|\})$
\end{itemize}

\end{enumerate}