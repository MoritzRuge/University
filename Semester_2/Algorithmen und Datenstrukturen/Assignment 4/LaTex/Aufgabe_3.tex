\section*{Problem 3: (a,b)-Bäume} 

\textbf{a)} Beschreiben Sie, wie man in einem (a, b)-Baum mit n Schlüsseln die Operation succ(k) implementieren kann. Was ist die Laufzeit?\\

$\Rightarrow$ succ(k) - finde den Nachfolge vom Schlüssel $k$

\begin{enumerate}
\item Suche den Knoten $"i"$, der den Eintrag $k$ enthält
\item Wenn $k$ nicht das größte Element in $u$ ist, schaue ob es noch ein Teilbaum zwischen dem Element $k$ und seinem direkten Nachfolger gibt.
	\begin{itemize}
		\item Wenn Nein $\rightarrow$ dann gib den direkten Nachfolger von $k$ zurück
		\item Wenn ja, gehe in den Teilbaum und gebe das kleinste Element zurück
	\end{itemize}
\item Wenn $k$ das größte Element in $u$ ist:
	\begin{itemize}
		\item wenn $i$ ein rechten Teilbaum besitzt, gehe in den rechten Teilbaum und gebe das kleinste Element zurück
		\item ansonsten gehe zum Elternknoten und suche das erste Element, das größer als $k$ ist und gebe es zurück
	\end{itemize}
\end{enumerate}

$\Rightarrow$ Die Laufzeit beträgt $O(logn)$, da die Höhe eines (a,b)-Baums $O(logn)$ ist. \\



\noindent
\textbf{b)} Beschreiben Sie, wie man in einem (a, b)-Baum mit n Schlüsseln die Operation $findRange(k1 , k2)$ implementieren kann, die alle Schlüssel $k$ liefert, für die
$k1 \leq k \leq k2$ ist. Die Laufzeit soll $O(b loga n + s)$ betragen. Dabei ist $s$ die Anzahl der gelieferten Schlüssel.\\

\noindent
\textbf{c)} Seien $T1$ und $T2$ zwei (a, b)-Bäume, und sei $S1$ die Schlüsselmenge von $T1$ und $S2$ die Schlüsselmenge von $T2$ . Sei $x$ ein weiterer Schlüssel. Alle Schlüssel in $S1$ sind kleiner als $x$, und alle Schlüssel in $S2$ sind größer als $x$. Beschreiben Sie eine Operation join, die aus $T1$, $T2$ und $x$ einen (a, b)-Baum für die Schlüsselmenge $S1 \cup {x} \cup S2$ erzeugt. Die Laufzeit sollte $O(b loga max{|S1|, |S2|})$ betragen. Hinweis: Betrachten Sie zunächst den Fall, dass $T1$ und $T2$ die gleiche Höhe haben. Achten Sie darauf, dass hinterher die (a, b)-Baum Eigenschaften wieder hergestellt werden.