\section*{{Problem 3: Hashing im Selbstversuch}} 

Fügen Sie nacheinander die Schlüssel 5, 28, 19, 15, 20, 33, 12, 17, 10 in eine Hashtabelle der Größe 9 ein. Die Hashfunktion sei $h(k) = k \mod 9$. Sie mit Die Konflikte werden mit Verkettung gelöst.\\
Illustrieren Sie jeweils die einzelnen Schritte. Wiederholen Sie dann mit der Hashfunktion $h(k) = k \mod 6$ und einer Hashtabelle der Größe 6.\\

\noindent
\textbf{Teilaufgabe 1: Hashtabelle der Größe 9}

\begin{multicols}{2}
\begin{itemize}
\item Schlüssel 5\\
Hashfunktion: $h(5) = 5 \mod 9=5$

\begin{center}
\begin{tabular}{c|c}
Index & Wert \\
\hline
0 & \\
1 & \\
2 & \\
3 & \\
4 & 5\\
5 & \\
6 & \\
7 & \\
8 & \\
\end{tabular}
\end{center}

\item Schlüssel 28\\
Hashfunktion: $h(28) = 28 \mod 9 = 1$

\begin{center}
\begin{tabular}{c|c}
Index & Wert \\
\hline
0 & \\
1 & 28\\
2 & \\
3 & \\
4 & \\
5 & 5\\
6 & \\
7 & \\
8 & \\
\end{tabular}
\end{center}

\item Schlüssel 19\\
Hashfunktion: $h(19) = 19 \mod 9 = 1$

\begin{center}
\begin{tabular}{c|c}
Index & Wert \\
\hline
0 & \\
1 & [28,19]\\
2 & \\
3 & \\
4 & \\
5 & 5\\
6 & \\
7 & \\
8 & \\
\end{tabular}
\end{center}

\item Schlüssel 15\\
Hashfunktion: $h(15) = 15 \mod 9 = 7$

\begin{center}
\begin{tabular}{c|c}
Index & Wert \\
\hline
0 & \\
1 & [28,19]\\
2 & 20\\
3 & \\
4 & \\
5 & 5\\
6 & \\
7 & 15\\
8 & \\
\end{tabular}
\end{center}

\item Schlüssel 20\\
Hashfunktion: $h(20) = 20 \mod 9 = 2$

\begin{center}
\begin{tabular}{c|c}
Index & Wert \\
\hline
0 & \\
1 & [28,19]\\
2 & 20\\
3 & \\
4 & \\
5 & 5\\
6 & \\
7 & 15\\
8 & \\
\end{tabular}
\end{center}

\item Schlüssel 33\\
Hashfunktion: $h(33) = 33 \mod 9 = 6$

\begin{center}
\begin{tabular}{c|c}
Index & Wert \\
\hline
0 & \\
1 & [28,19]\\
2 & 20\\
3 & \\
4 & \\
5 & 5\\
6 & 33\\
7 & 15\\
8 & \\
\end{tabular}
\end{center}

\item Schlüssel 12\\
Hashfunktion: $h(12) = 12 \mod 9 = 3$

\begin{center}
\begin{tabular}{c|c}
Index & Wert \\
\hline
0 & \\
1 & [28,19]\\
2 & 20\\
3 & 12\\
4 & \\
5 & 5\\
6 & 33\\
7 & 15\\
8 & \\
\end{tabular}
\end{center}

\item Schlüssel 17\\
Hashfunktion: $h(17) = 17 \mod 9 = 8$

\begin{center}
\begin{tabular}{c|c}
Index & Wert \\
\hline
0 & \\
1 & [28,19]\\
2 & 20\\
3 & 12\\
4 & \\
5 & 5\\
6 & 33\\
7 & 15\\
8 & 17\\
\end{tabular}
\end{center}

\columnbreak
\item Schlüssel 10\\
Hashfunktion: $h(10) = 10 \mod 9 = 1$

\begin{center}
\begin{tabular}{c|c}
Index & Wert \\
\hline
0 & \\
1 & [28,19,10]\\
2 & 20\\
3 & 12\\
4 & \\
5 & 5\\
6 & 33\\
7 & 15\\
8 & 17\\
\end{tabular}
\end{center}
\end{itemize}
\end{multicols}


\noindent
\textbf{Teilaufgabe 2: Hashtabelle der Größe 6}

\begin{multicols}{2}
\begin{itemize}
\item Schlüssel 5\\
Hashfunktion: $h(5) = 5 \mod 6=5$

\begin{center}
\begin{tabular}{c|c}
Index & Wert \\
\hline
0 & \\
1 & \\
2 & \\
3 & \\
4 & \\
5 & 5\\

\end{tabular}
\end{center}

\item Schlüssel 28\\
Hashfunktion: $h(28) = 28 \mod 6 = 4$

\begin{center}
\begin{tabular}{c|c}
Index & Wert \\
\hline
0 & \\
1 & \\
2 & \\
3 & \\
4 & 28\\
5 & 5\\

\end{tabular}
\end{center}

\item Schlüssel 19\\
Hashfunktion: $h(19) = 19 \mod 6 = 1$

\begin{center}
\begin{tabular}{c|c}
Index & Wert \\
\hline
0 & \\
1 & 19\\
2 & \\
3 & \\
4 & 28\\
5 & 5\\
\end{tabular}
\end{center}

\item Schlüssel 15\\
Hashfunktion: $h(15) = 15 \mod 6 = 3$

\begin{center}
\begin{tabular}{c|c}
Index & Wert \\
\hline
0 & \\
1 & 19\\
2 & \\
3 & 15\\
4 & 28\\
5 & 5\\
\end{tabular}
\end{center}

\item Schlüssel 20\\
Hashfunktion: $h(20) = 20 \mod 6 = 2$

\begin{center}
\begin{tabular}{c|c}
Index & Wert \\
\hline
0 & \\
1 & 19\\
2 & 20\\
3 & 15\\
4 & 28\\
5 & 5\\
\end{tabular}
\end{center}

\item Schlüssel 33\\
Hashfunktion: $h(33) = 33 \mod 6 = 3$

\begin{center}
\begin{tabular}{c|c}
Index & Wert \\
\hline
0 & \\
1 & 19\\
2 & 20\\
3 & [15,33]\\
4 & 28\\
5 & 5\\
\end{tabular}
\end{center}

\item Schlüssel 12\\
Hashfunktion: $h(12) = 12 \mod 6 = 0$

\begin{center}
\begin{tabular}{c|c}
Index & Wert \\
\hline
0 & 12\\
1 & 19\\
2 & 20\\
3 & [15,33]\\
4 & 28\\
5 & 5\\
\end{tabular}
\end{center}

\item Schlüssel 17\\
Hashfunktion: $h(17) = 17 \mod 6 = 5$

\begin{center}
\begin{tabular}{c|c}
Index & Wert \\
\hline
0 & 12\\
1 & 19\\
2 & 20\\
3 & [15,33]\\
4 & 28\\
5 & [5,17]\\
\end{tabular}
\end{center}

\columnbreak
\item Schlüssel 10\\
Hashfunktion: $h(10) = 10 \mod 6 = 4$

\begin{center}
\begin{tabular}{c|c}
Index & Wert \\
\hline
0 & 12\\
1 & 19\\
2 & 20\\
3 & [15,33]\\
4 & [28,10]\\
5 & [5,17]\\
\end{tabular}
\end{center}
\end{itemize}
\end{multicols}