\section*{{Problem 2: Hashing: Worst-case-Analyse}} 

Sei $A$ eine Hashtabelle der Größe $N$ , und $n \in N$ beliebig. Zeigen Sie: Für jede Schlüsselmenge $K$ mit $|K| \geq (n - 1)N + 1$ und jede Hashfunktion $h : K \rightarrow \{0,\dots, N - 1\}$ existiert eine Menge $S \subseteq K mit |S| = n$, so dass alle Elemente von $S$ auf denselben Eintrag in $A$ abgebildet werden. Was bedeutet das für die worst-case Laufzeit von Hashing mit Verkettung? Wie verträgt sich das mit der Analyse aus der Vorlesung?