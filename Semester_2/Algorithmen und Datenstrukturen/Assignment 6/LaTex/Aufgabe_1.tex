\section*{{Problem 1: Elementare Wahrscheinlichkeitsrechnung}}

Auf einem Tisch stehen $N$ Kisten. In diese Kisten werden nacheinander unabhängig voneinander $n$ Bälle geworfen, wobei jede Kiste mit gleicher Wahrscheinlichkeit getroffen wird.\\

\noindent
\textbf{a)} Berechnen Sie die Wahrscheinlichkeit, dass Kiste i leer ist. Sei Yi die Zufalls-
variable, die den Wert 1 annimmt, falls Kiste i leer ist, und 0 sonst. Geben Sie
auch den Erwartungswert E[Yi ] an.\\
Hinweis: Wenn Kiste $i$ leer bleibt, dann landen alle $n$ Bälle in den $N - 1$ anderen Kisten.\\

\noindent
\textbf{b)} Sei $X$ die Zufallsvariable, welche die Anzahl von leeren Kisten angibt. Berechnen Sie den Erwartungswert von $X$ mit Hilfe der Erwartungswerte $E[Yi]$.\\

\noindent
\textbf{c)} Bestimmen Sie, wie viele Bälle man benötigt, damit gilt: mit Wahrscheinlichkeit mindestens 1/2 enthält mindestens eine Kiste mindestens zwei Bälle.\\ 
$Hinweis:$ Stellen Sie sich vor, die $n$ Bälle werden nacheinander in die Kisten
geworfen. Was ist die Wahrscheinlichkeit, dass der nächste Ball in einer leeren Kiste landet, wenn alle vorherigen Bälle in einer leeren Kiste gelandet sind? Verwenden Sie die ungemein nützliche Abschätzung $1 + x \leq e^x$ , welche für alle $x \in R$ gilt.\\

\noindent
\textbf{d)} Professor Pinocchio hat eine Idee, um Hashtabellen zu vereinfachen. Wenn
wir die Zahl $N$ der Plätze in der Hashtabelle im Verhältnis zur Anzahl der zu
speichernden Einträge $n$ groß genug wählen, sollte die Wahrscheinlichkeit, dass
Kollisionen auftreten, verschwindend gering werden (unter der Annahme, dass
sich die Hashfunktion wie eine zufällige Funktion verhält). Dann könnte man
auf die Kollisionsbehandlung verzichten. In Anbetracht von (c), was halten Sie
von dem Vorschlag? (Wenn Sie Teil (c) nicht gelöst haben, dann bearbeiten Sie
diesen Teil unter der Annahme, dass für $N^{1/3}$ Bälle mindestens eine Kollision
auftritt.\\