%%%%% TODO %%%%%
% subsection a % Zwei Seiten layout ? damit die ganzen Bäume nicht so viel Platz wegnehmen

\section*{{Problem 1: AVL-Bäume}}


\subsection*{a) Fügen Sie die Schlüssel A, L, G, O, D, T, S, X, Y, Z in dieser Reihenfolge in einen anfangs leeren AVL-Baum ein. Löschen Sie sodann die Schlüssel Z, A, L. Zeichnen Sie den Baum nach jedem Einfüge- und Löschvorgang, und zeigen Sie die Rotationen, welche durchgeführt werden. Annotieren Sie dabei auch die Knoten mit ihrer jeweiligen Höhe.} 

$\Rightarrow$ Bei den Knoten die hochgestellte Zahl ist die Höhe des jeweiligen Knotens.

\begin{multicols}{2}

\begin{enumerate}
	\item Einfügen: A

\begin{tikzpicture}[scale=0.5,
  level 1/.style={sibling distance=40mm},
  level 2/.style={sibling distance=20mm}]
\node{$A^{0}$};
\end{tikzpicture}


\item Einfügen: L

\begin{tikzpicture}[scale=0.5,
  level 1/.style={sibling distance=40mm},
  level 2/.style={sibling distance=20mm}]
\node{$A^{1}$}
	child{node{$\Box^{-1}$}}
	child{node{$L^{0}$}};
\end{tikzpicture}

\item Einfügen: G

\begin{tikzpicture}[scale=0.5,
  level 1/.style={sibling distance=40mm},
  level 2/.style={sibling distance=20mm}]
\node{$A^{2}$}
	child{node{$\Box^{-1}$}}
	child{node{$L^{1}$}
		child{node{$G^{0}$}}
		child{node{$\Box^{-1}$}}};
\end{tikzpicture}

\begin{itemize}
	\item BF-Faktor bei Knoten $A$ ist größer als 1 $\Rightarrow$ Um-balancieren der Knoten $A,L,G$
\end{itemize}

$\Rightarrow$ Rechts-Rotation der Knoten $L\&G$

\begin{tikzpicture}[scale=0.5,
  level 1/.style={sibling distance=40mm},
  level 2/.style={sibling distance=20mm}]
\node{$A^{2}$}
	child{node{$\Box^{-1}$}}
	child{node{$G^{1}$}
		child{node{$\Box^{-1}$}}
		child{node{$L^{0}$}}};
\end{tikzpicture}

$\Rightarrow$ Links-Rotation der Knoten $A\&G$

\begin{tikzpicture}[scale=0.5,
  level 1/.style={sibling distance=40mm},
  level 2/.style={sibling distance=20mm}]
\node{$G^{1}$}
	child{node{$A^{0}$}}
	child{node{$L^{0}$}};
\end{tikzpicture}

$\Rightarrow$ AVL-Baum ist ausgeglichen

\item Einfügen: O

\begin{tikzpicture}[scale=0.5,
  level 1/.style={sibling distance=40mm},
  level 2/.style={sibling distance=20mm}]
\node{$G^{2}$}
	child{node{$A^{0}$}}
	child{node{$L^{1}$}
		child{node{$\Box^{-1}$}}
		child{node{$O^{0}$}}};
\end{tikzpicture}

\item Einfügen: D

\begin{tikzpicture}[scale=0.5,
  level 1/.style={sibling distance=40mm},
  level 2/.style={sibling distance=20mm}]
\node{$G^{2}$}
	child{node{$A^{1}$}
		child{node{$\Box^{-1}$}}
		child{node{$D^{0}$}}}
	child{node{$L^{1}$}
		child{node{$\Box^{-1}$}}
		child{node{$O^{0}$}}};
\end{tikzpicture}

\item Einfügen: T

\begin{tikzpicture}[scale=0.5,
  level 1/.style={sibling distance=40mm},
  level 2/.style={sibling distance=20mm}]
\node{$G^{3}$}
	child{node{$A^{1}$}
		child{node{$\Box^{-1}$}}
		child{node{$D^{0}$}}}
	child{node{$L^{2}$}
		child{node{$\Box^{-1}$}}
		child{node{$O^{1}$}
			child{node{$\Box^{-1}$}}
			child{node{$T^{0}$}}}};
\end{tikzpicture}

\begin{itemize}
	\item BF-Faktor bei Knoten $L$ ist größer als 1 $\Rightarrow$ Um-balancieren der Knoten $L,O,T$
\end{itemize}

$\Rightarrow$ Rechts-Rotation der Knoten $L\&O$

\begin{tikzpicture}[scale=0.5,
  level 1/.style={sibling distance=40mm},
  level 2/.style={sibling distance=20mm}]
\node{$G^{2}$}
	child{node{$A^{1}$}
		child{node{$\Box^{-1}$}}
		child{node{$D^{0}$}}}
	child{node{$O^{1}$}
		child{node{$L^{0}$}}
		child{node{$T^{0}$}}};
\end{tikzpicture}

$\Rightarrow$ AVL-Baum ist ausgeglichen

\item Einfügen: S

\begin{tikzpicture}[scale=0.5,
  level 1/.style={sibling distance=40mm},
  level 2/.style={sibling distance=20mm}]
\node{$G^{3}$}
	child{node{$A^{1}$}
		child{node{$\Box^{-1}$}}
		child{node{$D^{0}$}}}
	child{node{$O^{2}$}
		child{node{$L^{0}$}}
		child{node{$T^{1}$}
			child{node{$S^{0}$}}
			child{node{$\Box^{-1}$}}}};
\end{tikzpicture}

\newpage
\item Einfügen: X

\begin{tikzpicture}[scale=0.5,
  level 1/.style={sibling distance=40mm},
  level 2/.style={sibling distance=20mm}]
\node{$G^{3}$}
	child{node{$A^{1}$}
		child{node{$\Box^{-1}$}}
		child{node{$D^{0}$}}}
	child{node{$O^{2}$}
		child{node{$L^{0}$}}
		child{node{$T^{1}$}
			child{node{$S^{0}$}}
			child{node{$X^{0}$}}}};
\end{tikzpicture}

\item Einfügen: Y

\begin{tikzpicture}[scale=0.5,
  level 1/.style={sibling distance=40mm},
  level 2/.style={sibling distance=20mm}]
\node{$G^{4}$}
	child{node{$A^{1}$}
		child{node{$\Box^{-1}$}}
		child{node{$D^{0}$}}}
	child{node{$O^{3}$}
		child{node{$L^{0}$}}
		child{node{$T^{2}$}
			child{node{$S^{0}$}}
			child{node{$X^{1}$}
				child{node{$\Box^{-1}$}}
				child{node{$Y^{0}$}}}}};
\end{tikzpicture}

\begin{itemize}
	\item BF-Faktor bei Knoten $O$ ist größer als 1 $\Rightarrow$ Um-balancieren der Knoten $O\&T$
\end{itemize}

$\Rightarrow$ Links-Rotation der Knoten $O\&T$

\begin{tikzpicture}[scale=0.5,
  level 1/.style={sibling distance=60mm},
  level 2/.style={sibling distance=40mm},
  level 3/.style={sibling distance=20mm},
  level 4/.style={sibling distance=20mm}]
\node{$G^{3}$}
	child{node{$A^{1}$}
		child{node{$\Box^{-1}$}}
		child{node{$D^{0}$}}}
	child{node{$T^{2}$}
		child{node{$O^{1}$}
			child{node{$L^{0}$}}
			child{node{$S^{0}$}}}
		child{node{$X^{1}$}
			child{node{$\Box^{-1}$}}
			child{node{$Y^{0}$}}}};
\end{tikzpicture}

$\Rightarrow$ AVL-Baum ist ausgeglichen

\item Einfügen: Z

\begin{tikzpicture}[scale=0.5,
  level 1/.style={sibling distance=60mm},
  level 2/.style={sibling distance=40mm},
  level 3/.style={sibling distance=20mm},
  level 4/.style={sibling distance=20mm}]
\node{$G^{4}$}
	child{node{$A^{1}$}
		child{node{$\Box^{-1}$}}
		child{node{$D^{0}$}}}
	child{node{$T^{3}$}
		child{node{$O^{1}$}
			child{node{$L^{0}$}}
			child{node{$S^{0}$}}}
		child{node{$X^{2}$}
			child{node{$\Box^{-1}$}}
			child{node{$Y^{1}$}
				child{node{$\Box^{-1}$}}
				child{node{$Z^{0}$}}}}};
\end{tikzpicture}

\begin{itemize}
	\item BF-Faktor bei Knoten $X$ ist größer als 1 $\Rightarrow$ Um-balancieren der Knoten $Y\&Z$
\end{itemize}

$\Rightarrow$ Links-Rotation der Knoten $X\&Y$

\begin{tikzpicture}[scale=0.5,
  level 1/.style={sibling distance=60mm},
  level 2/.style={sibling distance=40mm},
  level 3/.style={sibling distance=20mm},
  level 4/.style={sibling distance=20mm}]
\node{$G^{3}$}
	child{node{$A^{1}$}
		child{node{$\Box^{-1}$}}
		child{node{$D^{0}$}}}
	child{node{$T^{2}$}
		child{node{$O^{1}$}
			child{node{$L^{0}$}}
			child{node{$S^{0}$}}}
		child{node{$Y^{1}$}
			child{node{$X^{0}$}}
			child{node{$Z^{0}$}}}};
\end{tikzpicture}

$\Rightarrow$ AVL-Baum ist ausgeglichen

\end{enumerate}
\end{multicols}

\subsection*{b) Beweisen Sie: Beim Einfügen in einen AVL-Baum wird höchstens eine (Einfach- oder Doppel-)Rotation ausgeführt. Gilt das auch beim Löschen (Begründung)?} 