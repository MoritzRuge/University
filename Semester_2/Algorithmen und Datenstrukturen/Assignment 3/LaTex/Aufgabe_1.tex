%%%%% TODO %%%%%
% subsection a % Zwei Seiten layout ? damit die ganzen Bäume nicht so viel Platz wegnehmen

\section*{{Problem 1: AVL-Bäume}}


\textbf{a)} Fügen Sie die Schlüssel A, L, G, O, D, T, S, X, Y, Z in dieser Reihenfolge in einen anfangs leeren AVL-Baum ein. Löschen Sie sodann die Schlüssel Z, A, L. Zeichnen Sie den Baum nach jedem Einfüge- und Löschvorgang, und zeigen Sie die Rotationen, welche durchgeführt werden. Annotieren Sie dabei auch die Knoten mit ihrer jeweiligen Höhe.\\

$\Rightarrow$ Bei den Knoten die hochgestellte Zahl ist die Höhe des jeweiligen Knotens.

\begin{multicols}{2}

\begin{enumerate}
	\item Einfügen: A

\begin{tikzpicture}[scale=0.5,
  level 1/.style={sibling distance=40mm},
  level 2/.style={sibling distance=20mm}]
\node{$A^{0}$};
\end{tikzpicture}


\item Einfügen: L

\begin{tikzpicture}[scale=0.5,
  level 1/.style={sibling distance=40mm},
  level 2/.style={sibling distance=20mm}]
\node{$A^{1}$}
	child{node{$\Box^{-1}$}}
	child{node{$L^{0}$}};
\end{tikzpicture}

\item Einfügen: G

\begin{tikzpicture}[scale=0.5,
  level 1/.style={sibling distance=40mm},
  level 2/.style={sibling distance=20mm}]
\node{$A^{2}$}
	child{node{$\Box^{-1}$}}
	child{node{$L^{1}$}
		child{node{$G^{0}$}}
		child{node{$\Box^{-1}$}}};
\end{tikzpicture}

\begin{itemize}
	\item BF-Faktor bei Knoten $A$ ist größer als 1 $\Rightarrow$ Um-balancieren der Knoten $A,L,G$
\end{itemize}

$\Rightarrow$ Rechts-Rotation der Knoten $L\&G$

\begin{tikzpicture}[scale=0.5,
  level 1/.style={sibling distance=40mm},
  level 2/.style={sibling distance=20mm}]
\node{$A^{2}$}
	child{node{$\Box^{-1}$}}
	child{node{$G^{1}$}
		child{node{$\Box^{-1}$}}
		child{node{$L^{0}$}}};
\end{tikzpicture}

$\Rightarrow$ Links-Rotation der Knoten $A\&G$

\begin{tikzpicture}[scale=0.5,
  level 1/.style={sibling distance=40mm},
  level 2/.style={sibling distance=20mm}]
\node{$G^{1}$}
	child{node{$A^{0}$}}
	child{node{$L^{0}$}};
\end{tikzpicture}

$\Rightarrow$ AVL-Baum ist ausgeglichen

\item Einfügen: O

\begin{tikzpicture}[scale=0.5,
  level 1/.style={sibling distance=40mm},
  level 2/.style={sibling distance=20mm}]
\node{$G^{2}$}
	child{node{$A^{0}$}}
	child{node{$L^{1}$}
		child{node{$\Box^{-1}$}}
		child{node{$O^{0}$}}};
\end{tikzpicture}

\item Einfügen: D

\begin{tikzpicture}[scale=0.5,
  level 1/.style={sibling distance=40mm},
  level 2/.style={sibling distance=20mm}]
\node{$G^{2}$}
	child{node{$A^{1}$}
		child{node{$\Box^{-1}$}}
		child{node{$D^{0}$}}}
	child{node{$L^{1}$}
		child{node{$\Box^{-1}$}}
		child{node{$O^{0}$}}};
\end{tikzpicture}

\item Einfügen: T

\begin{tikzpicture}[scale=0.5,
  level 1/.style={sibling distance=40mm},
  level 2/.style={sibling distance=20mm}]
\node{$G^{3}$}
	child{node{$A^{1}$}
		child{node{$\Box^{-1}$}}
		child{node{$D^{0}$}}}
	child{node{$L^{2}$}
		child{node{$\Box^{-1}$}}
		child{node{$O^{1}$}
			child{node{$\Box^{-1}$}}
			child{node{$T^{0}$}}}};
\end{tikzpicture}

\begin{itemize}
	\item BF-Faktor bei Knoten $L$ ist größer als 1 $\Rightarrow$ Um-balancieren der Knoten $L,O,T$
\end{itemize}

$\Rightarrow$ Rechts-Rotation der Knoten $L\&O$

\begin{tikzpicture}[scale=0.5,
  level 1/.style={sibling distance=40mm},
  level 2/.style={sibling distance=20mm}]
\node{$G^{2}$}
	child{node{$A^{1}$}
		child{node{$\Box^{-1}$}}
		child{node{$D^{0}$}}}
	child{node{$O^{1}$}
		child{node{$L^{0}$}}
		child{node{$T^{0}$}}};
\end{tikzpicture}

$\Rightarrow$ AVL-Baum ist ausgeglichen

\item Einfügen: S

\begin{tikzpicture}[scale=0.5,
  level 1/.style={sibling distance=40mm},
  level 2/.style={sibling distance=20mm}]
\node{$G^{3}$}
	child{node{$A^{1}$}
		child{node{$\Box^{-1}$}}
		child{node{$D^{0}$}}}
	child{node{$O^{2}$}
		child{node{$L^{0}$}}
		child{node{$T^{1}$}
			child{node{$S^{0}$}}
			child{node{$\Box^{-1}$}}}};
\end{tikzpicture}

\newpage
\item Einfügen: X

\begin{tikzpicture}[scale=0.5,
  level 1/.style={sibling distance=40mm},
  level 2/.style={sibling distance=20mm}]
\node{$G^{3}$}
	child{node{$A^{1}$}
		child{node{$\Box^{-1}$}}
		child{node{$D^{0}$}}}
	child{node{$O^{2}$}
		child{node{$L^{0}$}}
		child{node{$T^{1}$}
			child{node{$S^{0}$}}
			child{node{$X^{0}$}}}};
\end{tikzpicture}

\item Einfügen: Y

\begin{tikzpicture}[scale=0.5,
  level 1/.style={sibling distance=40mm},
  level 2/.style={sibling distance=20mm}]
\node{$G^{4}$}
	child{node{$A^{1}$}
		child{node{$\Box^{-1}$}}
		child{node{$D^{0}$}}}
	child{node{$O^{3}$}
		child{node{$L^{0}$}}
		child{node{$T^{2}$}
			child{node{$S^{0}$}}
			child{node{$X^{1}$}
				child{node{$\Box^{-1}$}}
				child{node{$Y^{0}$}}}}};
\end{tikzpicture}

\begin{itemize}
	\item BF-Faktor bei Knoten $O$ ist größer als 1 $\Rightarrow$ Um-balancieren der Knoten $O\&T$
\end{itemize}

$\Rightarrow$ Links-Rotation der Knoten $O\&T$

\begin{tikzpicture}[scale=0.5,
  level 1/.style={sibling distance=60mm},
  level 2/.style={sibling distance=40mm},
  level 3/.style={sibling distance=20mm},
  level 4/.style={sibling distance=20mm}]
\node{$G^{3}$}
	child{node{$A^{1}$}
		child{node{$\Box^{-1}$}}
		child{node{$D^{0}$}}}
	child{node{$T^{2}$}
		child{node{$O^{1}$}
			child{node{$L^{0}$}}
			child{node{$S^{0}$}}}
		child{node{$X^{1}$}
			child{node{$\Box^{-1}$}}
			child{node{$Y^{0}$}}}};
\end{tikzpicture}

$\Rightarrow$ AVL-Baum ist ausgeglichen

\item Einfügen: Z

\begin{tikzpicture}[scale=0.5,
  level 1/.style={sibling distance=60mm},
  level 2/.style={sibling distance=40mm},
  level 3/.style={sibling distance=20mm},
  level 4/.style={sibling distance=20mm}]
\node{$G^{4}$}
	child{node{$A^{1}$}
		child{node{$\Box^{-1}$}}
		child{node{$D^{0}$}}}
	child{node{$T^{3}$}
		child{node{$O^{1}$}
			child{node{$L^{0}$}}
			child{node{$S^{0}$}}}
		child{node{$X^{2}$}
			child{node{$\Box^{-1}$}}
			child{node{$Y^{1}$}
				child{node{$\Box^{-1}$}}
				child{node{$Z^{0}$}}}}};
\end{tikzpicture}

\begin{itemize}
	\item BF-Faktor bei Knoten $X$ ist größer als 1 $\Rightarrow$ Um-balancieren der Knoten $Y\&Z$
\end{itemize}

$\Rightarrow$ Links-Rotation der Knoten $X\&Y$

\begin{tikzpicture}[scale=0.5,
  level 1/.style={sibling distance=60mm},
  level 2/.style={sibling distance=40mm},
  level 3/.style={sibling distance=20mm},
  level 4/.style={sibling distance=20mm}]
\node{$G^{3}$}
	child{node{$A^{1}$}
		child{node{$\Box^{-1}$}}
		child{node{$D^{0}$}}}
	child{node{$T^{2}$}
		child{node{$O^{1}$}
			child{node{$L^{0}$}}
			child{node{$S^{0}$}}}
		child{node{$Y^{1}$}
			child{node{$X^{0}$}}
			child{node{$Z^{0}$}}}};
\end{tikzpicture}

$\Rightarrow$ AVL-Baum ist ausgeglichen

\item Lösche: Z

\begin{tikzpicture}[scale=0.5,
  level 1/.style={sibling distance=60mm},
  level 2/.style={sibling distance=40mm},
  level 3/.style={sibling distance=20mm},
  level 4/.style={sibling distance=20mm}]
\node{$G^{3}$}
	child{node{$A^{1}$}
		child{node{$\Box^{-1}$}}
		child{node{$D^{0}$}}}
	child{node{$T^{2}$}
		child{node{$O^{1}$}
			child{node{$L^{0}$}}
			child{node{$S^{0}$}}}
		child{node{$Y^{1}$}
			child{node{$X^{0}$}}
			child{node{$\Box^{-1}$}}}};
\end{tikzpicture}

\item Lösche: A

\begin{tikzpicture}[scale=0.5,
  level 1/.style={sibling distance=60mm},
  level 2/.style={sibling distance=40mm},
  level 3/.style={sibling distance=20mm},
  level 4/.style={sibling distance=20mm}]
\node{$G^{3}$}
	child{node{$D^{0}$}
		child{node{$\Box^{-1}$}}
		child{node{$\Box^{-1}$}}}
	child{node{$T^{2}$}
		child{node{$O^{1}$}
			child{node{$L^{0}$}}
			child{node{$S^{0}$}}}
		child{node{$Y^{1}$}
			child{node{$X^{0}$}}
			child{node{$\Box^{-1}$}}}};
\end{tikzpicture}

\begin{itemize}
	\item BF-Faktor bei Knoten $G$ ist größer als 1 $\Rightarrow$ Um-balancieren der Knoten $G\&T$
\end{itemize}

\begin{tikzpicture}[scale=0.5,
  level 1/.style={sibling distance=60mm},
  level 2/.style={sibling distance=40mm},
  level 3/.style={sibling distance=20mm},
  level 4/.style={sibling distance=20mm}]
\node{$T^{3}$}
	child{node{$G^{2}$}
		child{node{$A^{0}$}}
		child{node{$O^{1}$}
			child{node{$L^{0}$}}
			child{node{$S^{0}$}}}}
	child{node{$Y^{1}$}
		child{node{$X^{0}$}}
		child{node{$\Box^{-1}$}}};
\end{tikzpicture}

$\Rightarrow$ AVL-Baum ist ausgeglichen

\item Lösche: L

\begin{tikzpicture}[scale=0.5,
  level 1/.style={sibling distance=60mm},
  level 2/.style={sibling distance=40mm},
  level 3/.style={sibling distance=20mm},
  level 4/.style={sibling distance=20mm}]
\node{$T^{3}$}
	child{node{$G^{2}$}
		child{node{$A^{0}$}}
		child{node{$O^{1}$}
			child{node{$\Box^{-1}$}}
			child{node{$S^{0}$}}}}
	child{node{$Y^{1}$}
		child{node{$X^{0}$}}
		child{node{$\Box^{-1}$}}};
\end{tikzpicture}

\end{enumerate}
\end{multicols}

\newpage
\noindent
\textbf{b)} Beweisen Sie: Beim Einfügen in einen AVL-Baum wird höchstens eine (Einfach- oder Doppel-)Rotation ausgeführt. Gilt das auch beim Löschen (Begründung)?\\


\noindent
\textbf{Annahme:} Beim Einfügen in einen AVL-Baum wird höchstens eine (Einfach- oder Doppel-)Rotation ausgeführt.
\begin{itemize}
	\item In einem AVL-Baum gelten die Eigenschaften eines BTS.
	\item In einem AVL-Baum muss gelten: Die Differenz zwischen der Höhe des Linken Teilbaum und der Höhe des Rechten Teilbaum darf maximal 1 sein.
\end{itemize}

\noindent
base-case AVL-Baum mit einem Knoten: \\
r $\rightarrow$ h: 0 \\

\noindent
insertion (n):
\begin{verbatim}
```
       r 	h: 0
     /   \
    n     nil 	hl: 1, hr: 0
```
\end{verbatim}

$|hl - hr| <= 1$ true $\rightarrow$ keine Rotation nötig!\\
\textbf{Annahme gilt im base-case}\\

\textbf{I.S:} Wenn die Annahme bei n Insertions gilt, gilt sie auch n+1 Insertions. Wenn nach n insertions ein AVL-Baum die AVL-Eigenschaften weiterhin erfüllt kann es zu mehreren Fallen bei einer $n+1$ insertion kommen:\\
Fallunterscheidung:
\begin{enumerate}
	\item Nach einer Insertion wird kein Knoten unblanciert. $\rightarrow$ keine Rotation notwendig
	\item Nach einer Insertion wird ein Knoten unblanciert, da einer der Teilbäume tiefer ist als der andere und $|hl - hr| <= 1$  nicht mehr gilt.
Da die Unbalance nur nach einer einzelnen Insertion auftreten kann, ist der Höhen Unterschied immer 2.
	\item[$2.1.$] der Linke Teilbaum eines Knotens ist Tiefer als der Rechte Teilbaum.
\end{enumerate}

\begin{verbatim}
``` Beispielabschnitt für links Unbalance irgendwo in einem AVL-Baum
       n
     /   \
    n1    n1
   /
  n2
 /
n3
```
\end{verbatim}

Am tiefsten unbalanciertem Knoten (pivot) wird eine rechts-Rotation ausgeführt.
\begin{itemize}
	\item Die Rotation wird am tieften unbalanciertem Knoten ausgeführt, weil nach einer Rotation die Höhe des betroffenen Teilbaums wieder den selben Wert hat, wie vor der n+1 Insertion.
	\item Der Elternknoten des pivot-Knotens wird zum neuen Rechten Kind des pivot-Knotens.
	\item Das alte Rechte Kind des pivot-Knotens wird zum neuen linken Kind des neuen Rechten Kindknotens.
\end{itemize}

\begin{verbatim}
```
      n
    /   \
   n1    n1
  /  \
 n2   n2
```
\end{verbatim}

\begin{itemize}
	\item[2.2] Der Rechte Teilbaum eines Knotens ist Tiefer als der Linke Teilbaum. Zeichnungen sind analog.\\
Am tiefsten unbalanciertem Knoten (pivot) wird eine links-Rotation ausgeführt. 
	\begin{itemize}
		\item Die Rotation wird am tieften unbalanciertem Knoten ausgeführt, weil nach einer Rotation die Höhe des betroffenen Teilbaums wieder den selben Wert hat, wie vor der $n+1$ Insertion.
		\item Der Elternknoten des pivot-Knotens wird zum neuen Linken Kind des pivot-Knotens.
		\item Das alte Linke Kind des pivot-Knotens wird zum neuen Rechten Kind des neuen Linken Kindknotens.
	\end{itemize}
	\item[2.3] Der Rechte-Linke Teilbaum eines Knotens ist Tiefer als der Rechte Teilbaum.
Es wird erst eine links-Rotation am tieften unbalanciertem Knoten (als Pivot Elternknoten) durchgeführt und dann eine weitere rechts-Rotation
mit dem selben pivot-Knoten durchgeführt. 
	\begin{itemize}
		\item Die Rotationen selbst funktionieren genau so wie in 2.1 und 2.2 beschrieben und sequentiell am selben pivot-Knoten.
		\item Auch bei Doppel-Rotationen gilt: Nach der Rotation sind die Höhenverhältnisse der Teilbaume wieder so wie vor der Insertion $n+1$
	\end{itemize}
\end{itemize}

\begin{verbatim}
``` Beispielabschnitt für rechts-links Unbalance irgendwo in einem AVL-Baum
        n
      /   \
    n1     n1
      \
      n2
     /
    n3
```
nach links-Rotation
```
        n
      /   \
    n1     n1
      \
      n2
        \
         n3
```
nach rechts-Rotation
```
        n
      /   \
    n1     n1
   /  \
  n2  n2
```
\end{verbatim}

\begin{itemize}
	\item[2.4] Der Linke-Rechte Teilbaum eines Knotens ist Tiefer als der Linke Teilbaum. Zeichnungen sind analog. Es wird erst eine rechtss-Rotation am tieften unbalanciertem Knoten (als Pivot Elternknoten) durchgeführt und dann eine weitere links-Rotation mit dem selben pivot-Knoten durchgeführt.
	
	\begin{itemize}
		\item Die Rotationen selbst funktionieren genau so wie in 2.1 und 2.2 beschrieben und sequentiell am selben pivot-Knoten.
		\item Auch bei Doppel-Rotationen gilt: Nach der Rotation sind die Höhenverhältnisse der Teilbaume wieder so wie vor der Insertion $n+1$
	\end{itemize}
\end{itemize}

Man sieht man benötigt maximal eine (Einfach- oder Doppel-) Rotation um nach einer Insertion die AVL-Eigenschaft zu erhalten.\\
Annahme gilt für $n+1$
\begin{flushright}
	$\Box$
\end{flushright}


\textbf{Löschen:}
\begin{itemize}
	\item Angenommen man hat einen beliebigen AVL-Baum. 
	\item Beim Löschen eines beliebigen Knotens kann es sein, dass eine komplexere Unblance eintritt, diese ist nicht durch einmaliges rotieren lösbar.
\end{itemize}
\begin{verbatim}
 bsp:
```
          20
      /        \
     15         25
    /  \       /  \
   10   18    24   30
       /  \
      17   19
```
\end{verbatim}

wenn man 15 entfernen würde, wäre die entstehende Unbalance nicht mit einer einmaligen Rotation lösbar. Daher gilt die Annahme meiner Meinung nach nur für das Einfügen (Insertion)