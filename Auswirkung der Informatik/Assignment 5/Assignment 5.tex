\documentclass[a4paper]{assignment}
\coursetitle{Auswirkung der Informatik}
\courselabel{AdI}
\exercisesheet{Assignment 5}{Digitale Transformation}
\student{Moritz Ruge \& Amelie Dzierzawa}
\school{Bachelor Informatik}
\university{Freie Universitäten Berlin}
\semester{Winter 2024}
\date{17 November 2024}
%\usepackage[pdftex]{graphicx}
%\usepackage{subfigure} 
\usepackage[backend=biber,style=numeric,url=true]{biblatex} % für Quellenangaben und Bibliotheken
\usepackage[]{hyperref} % für hyperlinks
\usepackage{array} % für die Tabelle
\usepackage{xcolor} %für Farben bei Text
\addbibresource{ref.bib}  % Add your .bib file here

\hypersetup{
    colorlinks=true,
    linkcolor=blue,
    filecolor=magenta,      
    urlcolor=cyan,
    pdftitle={Overleaf Example},
    pdfpagemode=FullScreen,
    }

\begin{document}

% Aufgabe C)
\begin{problemlist}

%Probelmnummer und Frage
\pbitem Recherchieren Sie: Welche Interessen vergolgt der CCC als Verein?

%Beginn der Antwort
\begin{answer}

%Beantwortung der Recherche
Präambel des CCC:\cite{ccc-satzung}
\begin{itemize}
\item Menschenrecht auf ungehinderter Kommunikation
\item Informationsfreiheit für jedermann
\item Beschäftigung mit der Auswirkung von Technologie aud die Gesellschaft und auf das Individum
\item Förderung von Wissen
\end{itemize}

Der CCC hat folgende Motivationen und Grenzen.\\
Hackerethik:\cite{ccc-hackerethik}
\begin{itemize}
\item Der Zugang zu Computern und allem, was einem zeigen kann, wie diese Welt funktioniert, sollte unbegrenzt und vollständig sein.
\item Alle Informationen müssen frei sein.
\item Misstraue Autoritäten – fördere Dezentralisierung.
\item Beurteile einen Hacker nach dem, was er tut, und nicht nach üblichen Kriterien wie Aussehen, Alter, Herkunft, Spezies, Geschlecht oder gesellschaftliche Stellung.
\item Man kann mit einem Computer Kunst und Schönheit schaffen.
\item Computer können dein Leben zum Besseren verändern.
\item Mülle nicht in den Daten anderer Leute.
\item Öffentliche Daten nützen, private Daten schützen.
\end{itemize}

% Ende Aufgabe C)
\end{answer}



% Aufgabe D+E)

%Probelmnummer und Frage
\newpage
\pbitem Führen Sie eine Analyse der Interessen folgender Beteiligten zur Digitalisierung des deutschen Gesundheitssystems durch \& Ordnen Sie den Interessen Werte zu, die diesen zugrunde liegen. Beispielweise: ‘Gerechtigkeit’, ‘Transparenz’. Finden Sie möglichst andere oder spezifischere Werte:

%begin der Problemaufzählung
\begin{itemize}

\item Interessen der BürgerInnen Deutschlands. Beispielsweise: ‘Verbesserung der Versorgung im Notfall’

\item Interessen des CCC (mit Bezug auf den Vortrag sowie weiterer Quellen) Beispielsweise: ‘Öffentlichkeit informieren’

\item Interessen der Bundesregierung Beispielweise: ‘Zustimmung durch WählerInnenfinden
\end{itemize}

Suchen sie jeweils andere Interessen, als die oben genannten.
Finde sie Belege für Ihre Aussagen, alle Interessen sollten sich klar auf Aussagen oder
Dokumente der Akteure (oder deren Repräsentanten) zurück verfolgen lassen.

%begin von Aufgabe d)

\begin{enumerate}

% Interesse von Ärzte
\item Interesse der Ärzte\cite{ccc-epa}

%Argument 1
\begin{itemize}
\item Datenübergabe in die ePA sollte ohne weitere "Mausklicks" vollzogen werden.\textcolor{red}{[Verbesserung]}

\begin{itemize}
\item Neben der ePA gibt es von Ärzten ein lokale Arztakte, ePA und das lokale system müssen sich immer wieder austauschen $\rightarrow$ mehr Aufwand

\item Nicht möglich, die Datenübertragung geschieht laut Kassenärztlicher Bundesvereinigung(KBV) nicht Automatisch, muss Manuel übertragen wer.(von Arzt) $\rightarrow$ mehr Aufwand

\item ePa vorhanden, Arzt muss Aufklären welche Metadaten gespeichert werden $\rightarrow$ mehr Aufwand
\end{itemize}

%Argument 2 - Patient hat kein ePA
\item Patient muss vor der Behandlung gefragt werden, ob er ePA benutzt - kein ePA oder nicht möglich \textcolor{red}{[Verbesserung]}
\begin{itemize}
\item Risikoaufklärung, Dokumentation der Aufklärung vom Arzt nötig
\item andere Ärzte hätten dann kein Zugriff auf die vollzogene Behandlung $\rightarrow$ mögliche Behandlungsfehler von anderen Ärzten.
\end{itemize}

%Argument 3 - Besondere Fälle
\item Besondere Fälle mit stigmatisierender Wirkung \textcolor{red}{[Verbesserung]}

\begin{itemize}
\item bei Geschlechtserkrankungen $\rightarrow$ Aufklärung und Dokumentation, mehr Aufwand
\item bei Schwangerschaftsabbrüchen $\rightarrow$ Aufklärung und Dokumentation, mehr Aufwand
\item bei psychischen Erkrankungen $\rightarrow$ Aufklärung und Dokumentation, mehr Aufwand
\end{itemize}

%Argument 4 - Nicht Erwachsen
\item Jugendliche und ePA \textcolor{red}{[Verbesserung]}

\begin{itemize}
\item Bei Jugendlichen ist Automatisierung gar nicht möglich$\rightarrow$ Riesiger mehr Aufwand
\begin{itemize}
	\item "Problem der geteilten Wissenssphären"
	\begin{itemize}
		\item Welche vom Minderjährigen erlangten Informationen über den Minderjährigen dürfen Eltern sehen?
		\item Welche vom Minderjährigen erlangten Informationen über die Eltern dürfen Eltern sehen?
	\end{itemize}
\end{itemize}
\end{itemize}

%Zsmfassung
\newpage
\begin{answer}
\textbf{Entweder} kein zusätzlicher Aufwand (\textcolor{red}{automatisierte Datenübertragung}) der Arztakte $\rightarrow$ systematischer Verstoß der Ärzte gegen Aufklärungs- \& Dokumentationspflichten (BGB), die Löschpflicht (DSGVO) und die ärztliche Schweigepflicht (StGB) \\
\textbf{Oder Checkliste} Aufklärung und Dokumentation, \textbf{Kuratieren der Daten} auf Daten über Dritte und löschpflichtige Daten vor der Übertragung zu ePA \\
$\rightarrow$ ePA-Befüllung brauch mehr Zeit als die Patientenversorgung!
\end{answer}
\end{itemize}


% Interesse von Bürgern

\item Interesse von Bürgern:

\begin{itemize}

%Sicherheit
\item Sicherheit
\begin{itemize}
	\item Daten sollten von niemanden außer dem Endnutzer und Arzt einsehbar sein.
	\item Hohe Verschlüsselung der Daten
\end{itemize}

%Transparenz
\item Transparenz
\begin{itemize}
	\item Leichter Zugang zu all meinen persönlichen Daten
\end{itemize}

%Kontrolle
\item Kontrolle
\begin{itemize}
	\item Der Endnutzer hat volle Kontrolle über seine Daten und das Recht auf Löschung dieser
\end{itemize}

%Information leicht zugänglich
\item Zugänglichkeit
\begin{itemize}
	\item Keine allzu verwirrenden Apps, einfache Strukturierung
	\item möglichst einfache Vorgänge, trotz hoher Sicherheit
	\item Daten immer und überall abrufbar
\end{itemize}

%Verbesserung der Notfallversorgung
\item Verbesserung
\begin{itemize}
	\item Verbesserung der Notfallversorgung
\end{itemize}

%Verringerung der Kosten
\item Kosten/Gelder
\begin{itemize}
	\item Verringerung der Krankenkassenbeiträge bzw. besserer Ausgleich
\end{itemize}
\end{itemize}


%Interessen des CCC:

\item Interessen des CCC:

\begin{itemize}

%Transparenz
\item Transparenz
\begin{itemize}
	\item Vollständige Aufklärung der Öffentlichkeit in einfacher Sprache
\end{itemize}

%Sicherheit
\item Sicherheit
\begin{itemize}
	\item Verbesserung der Sicherheit, bevor man Digitalisiert
\end{itemize}
\end{itemize}

% Interessen der Bundesregierung

\item Interessen der Bundesregierung:\cite{bgm-digitalisierung-im-gesundheitswesen}
\begin{itemize}

%Verbesserung
\item Verbesserung
\begin{itemize}
	\item Schnellere Kommunikation
	\item Schnellere Verwaltungsabläufe
	\item Abschaffung von Fax- \& Papiernachrichten
	\item Möglichkeit der systematischen Auswertung von Patientendaten
	\begin{itemize}
		\item Früherkennung von Krankheiten
		\item Neue Heilungschancen
		\item Individuell ausgerichtete Therapien
	\end{itemize}
\end{itemize}

%Zugänglichkeit
\item Zugänglichkeit
\begin{itemize}
	\item Bereitstellung von Patientendaten immer und überall
\end{itemize}
\end{itemize}

% Ende der Aufzählung von Aufgabe D+E

\end{enumerate}

% Beginn der Antwort Aufgabe D+E

\begin{answer}
\begin{itemize}
\item "Die Technik ist für den medizinischen Alltag nahezu ungeeignet."(...). "Sie raubt Zeit. Und Ärzte haben keine Zeit." -Peter Gocke Digitalchef der Charité
\item Bundesregierung möchte vorwiegend eine Verbesserung im Verwaltungsablauf für sich und die Bürger
\item CCC möchte eine Verbesserung in der Sicherheit und Transparenz
\item Bürger möchte Verbesserungen in Sachen Sicherheit, Transparenz, Kontrolle, Zugänglichkeit und Kosten/Gelder
\end{itemize}
\end{answer}

%Ende Aufgabe D+E)



%Aufgabe f)
\pbitem Identifizieren Sie Konflikte zwischen den verschiedenen Interessen und Wertvorstellungen. Begründen Sie.

%Tabelle
\begin{center}
\begin{tabular}{| m{3.5cm} | m{10cm} |}
\hline
Konflikt & Begründung \\
\hline
Gelder$\leftrightarrow$Verbesserung & Verbesserung bringt meist Kosten mit sich, vor allem wenn es sich um technologische Verbesserungen handelt \\
\hline
Sicherheit & Bei der Bemühung um digitale Transparenz könnten leicht Lücken in der Cybersecurity entstehen \\
\hline
\end{tabular}
\end{center}

%Aufgabe g)
\pbitem Sind die Interessen der BürgerInnen Deutschlands bei der ePA-Einführung gut bedient worden? Wo ja? Wo nein? Insgesamt? Begründen Sie.

\begin{itemize}
	\item Sicherheit
	\begin{itemize}
		\item Unzureichend
		\begin{itemize}
			\item Heilberufskarten zu leicht zugänglich
		\end{itemize}
	\end{itemize}		
	
	\item Transparenz
	\begin{itemize}
		\item Ja
	\end{itemize}
	\item Kontrolle
	\begin{itemize}
		\item Ja
	\end{itemize}
	\item Gelder
	\begin{itemize}
		\item $\rightarrow$ nicht wichtiges Kriterium für ePA
	\end{itemize}	
	\item Zugänglichkeit
	\begin{itemize}
		\item Ja
	\end{itemize}	
	\item Verbesserung
	\begin{itemize}
		\item Ja
	\end{itemize}	
	\item Kontrolle
	\begin{itemize}
		\item Ja
	\end{itemize}	
	\item Insgesamt
	\begin{itemize}
		\item Nein
		\begin{itemize}
			\item Sicherheit, der größte Faktor, wurde meiner Meinung nach nicht ausreichend bearbeitet
		\end{itemize}
	\end{itemize}
\end{itemize}

\end{problemlist}

%Quellen Reference
\newpage
\printbibliography

\end{document}
