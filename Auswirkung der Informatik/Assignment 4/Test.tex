\documentclass[11pt,a4paper]{article}
\usepackage[utf8]{inputenc}
\usepackage{amsmath}
\usepackage{amsfonts}
\usepackage{amssymb}
\usepackage[]{hyperref}
\usepackage[backend=biber,style=numeric,url=true]{biblatex}
\addbibresource{ref.bib}  % Add your .bib file here

\hypersetup{
    colorlinks=true,
    linkcolor=blue,
    filecolor=magenta,      
    urlcolor=cyan,
    pdftitle={Overleaf Example},
    pdfpagemode=FullScreen,
    }


\author{Amelie Dzierzawa \& Moritz Ruge}
\title{Assignment 4 - Flugzeugabsturz}
\begin{document}
\maketitle
\newpage

\section*{Aufgabe 4 - 1: Recherche - Boeing Absturz}
\subsection*{a) Maßnahmen:}

\begin{center}
\begin{itemize}
\item Startverbot nach dem 2. Absturz von der FAA
\item Entwicklung einer neuen Software-Architektur für Flugkontrollsysteme der Boeing 737 max (Fail-Safe-Prinzip $\longrightarrow$  mehr Sicherheit)
\end{itemize}
\end{center}

\subsection*{Ursachen:}
\begin{itemize}
\item Stabilisierungssoftware MCAS(Maneuvering Characteristics Augmentation System) (soll Flugzeug vor Stromabrissretten) war fehlerhaft: hat Nase von Flugzeug nach unten gedrückt weil es falsche Daten von defektem Anstellwinkelsensor bekam
\item 2 Bordcomputer kontrollieren sich nicht gegenseitig sondern abwechselnd
\item Gerücht über Softwarefehler: Warnlicht funktioniert nicht $\longrightarrow$ sollte anzeigen wenn 2 Anstellwinkelsensoren unterschiedliche Daten anzeigen
\item Co-Pilot hat Anweisungen des Piloten sehr langsam und unzureichend umgesetzt
\item Schon zuvor im Training durch Schwächen aufgefallen
\end{itemize}

\textbf{Mitte 2024 kommt raus: }

\begin{flushleft}
Boeing bekennt sich schuldig, US-Regierung betrogen zu haben um Gerichtsprozess der 2 Abstürze zu entkommen (Haben gegen Compliance- und Ethik-programm verstoßen, trotz Deal)
\end{flushleft}

\subsection*{b): Warum entwickelt? Was sind die Unterschiede?}
\begin{itemize}
\item Weiterentwicklung des Vorgägners
\begin{itemize}
\item Andere Triebwerke
\item Mehr Platz
\item Größere Effizienz ( Aerodynamische Änderungen )
\end{itemize}
\item Wurde als Kurzstreckenflugzeug Entwickelt
\item Sollte an Airbus verlorene Markanteile aufholen
\end{itemize}

\subsection*{Ziele:}
\begin{itemize}
\item Entwicklung eines Flugzeuges, dass mit der Konkurrenz (Airbus - A320neo) mithalten kann\cite{B-Wikipedia}
\begin{itemize}
	\item Reduzierung des Treibstoffverbrauchs um 15\% (wie vom A320neo)
	\item Erste Schätzung lagen bei 10-12\%, später bei 14,5\%
\end{itemize}
\item MCAS
\begin{itemize}
\item Für alle ungewöhnlichen Fluglagen
\item Vergleich der Messdaten zweier Sensoren zur Erfassung des 			Anstellwinkels
\item MCAS läst nur aus wenn beide Daten übereinstimmen
\begin{itemize}
\item Kann nur einmal ausgelösen
\end{itemize}
\item MCAS kann Steuereingabe über Steuerhorn des Piloten nicht mehr außer Kraft setzten
\end{itemize}
\item Vorher:
\begin{itemize}
\item MCAS bekam Daten nur von einem einzelnen Sensor
\item Auslösung wiederholte sich immer wieder $\longrightarrow$ Sensor meldete erneut erhöhten Anstellwinkel
\end{itemize}
\item Neu:
\begin{itemize}
\item Beide Sensoren liefern die Daten an MCAS
\begin{itemize}
\item Und beide Datensätze müssen übereinstimmen
\end{itemize}
\end{itemize}
\item Verhältnis zu Manual Electric Trim Wheel:
\begin{itemize}
\item Switch wurde ausgeschaltet
\begin{itemize}
\item Trim Wheel konnte nicht mehr genutzt werden
\end{itemize}
\item Switch wurde wieder angeschaltet $\longrightarrow$ MCAS wurde wieder Aktiviert
\item Berücksichtigte Datenquelle durch MCAS:
\begin{itemize}
\item Nur einer von zwei Sensoren lieferte Daten an MCAS
\end{itemize}
\item Rolle von MCAS in Abstürze:
\begin{itemize}
\item MCAS hat nur von einen Sensor Daten erhalten
\item Sensor war defekt, da druch den Anstellwinkel falsch berechnet
\item MCAS hat die Nase des Flugzeugs immer wieder nach Unten gedrückt um den Anstellwinkel zu "Korrigieren"
\end{itemize}
\end{itemize}
\end{itemize}

\newpage
\subsection*{d) FAA und der Absturz}
\begin{itemize}
\item FAA
\begin{itemize}
\item Federal Abiation Administration
\item Bundesluftfahrtbehörde der USA
\item Regulierung der zivilen Luftfahrt und des kommerziellen Lufttransports in den USA
\item Nach den beiden Abstürzen wurde die Zertifizierung von der FAA im Jahre 2020 überholt\cite{Boeing-737-MAX}
\begin{itemize}
\item Firmen müssen bei Design/- oder Änderungen von Flugeingenschaften, einen neuen Antrag stellen
\end{itemize} 
\end{itemize}
\item Zertifizierungsprozess
\begin{itemize}
\item 4 Flugzeuge intensiv geprüft
\item Sollte als Folgeversion des 737NG(Next Generation) zertifiziert werden (Supplemental type certificate – STC)
\item FAA hat Boeing erlaubt für die FAA Sicherheitsanalysen anzufertigen\cite{B-flawd-analysis}
\begin{itemize}
\item Analyse wurde für die Zertifizierung benutzt ( auch in der EU )
\end{itemize}
\item es muss weniger neu Zertifiziert werden $\longrightarrow$ Pilotenausbildung muss nicht angepasst werden, bedarf an Pilotenausbildung wird reduziert $\longrightarrow$ weniger Kosten
\end{itemize}

	\item MCAS FAA
	\begin{itemize}
		\item MCAS wurde nicht auf Fehler geprüft (außer von Boeing selbst), da es nicht nötig war von den Protokollen her
	\end{itemize}
	\item MCAS Einstufung
	\begin{itemize}
		\item Major
		\item Keine Fehlerbaumanalyse
		\item Bewertung der Fehler erforderte keine wietere Analyse
	\end{itemize}
\end{itemize}

\newpage
\section*{4-2}
\subsection*{Gründe \& Urgründe}
\begin{itemize}
	\item Unzureichende Überprüfung von MCAS
	\item Analyse von Sicherheitsrelevanten Systeme von Boeing selbst
	\item Zu wenig Sensoren
	\item Fehlerhafte Architektur in MCAS
	\item Unzureichende Ausbildung und Schulung der Piloten
\end{itemize}

\subsection*{Verantwortung}
\begin{itemize}
	\item FAA hätte besser kontrollieren müssen
	\item Co-Pilot hätte sich selbst besser einschätzen können
	\item Architekten und Techniker des Flugzeugs
	\item Manager von Boeing, die auf höhere Gewinnmaximierung aus sind
\end{itemize}

\subsection*{Warum hat der Zertifizierungsprozess nicht geleistet was er sollte?}
\begin{itemize}
	\item FAA hat nicht genau genug kontrolliert $\longrightarrow$ MCAS wurde nicht analysiert
	\item Boeing hat Sicherheitsrelevante Informationen unterschlagen um Geld zu Sparen (STC)
\end{itemize}

\subsection*{Vorbeugung}
\begin{itemize}
	\item Bessere Kontrollen
	\item Bessere Schulung des Personals
	\item Sicherheitsrelevante Überprüfungen von Öffentlichen Organisationen
\end{itemize}

\printbibliography

\href{https://www.deutschlandfunk.de/nach-abstuerzen-der-boeing-737-max-us-luftfahrtbehoerde-im-100.html}{Nach Abstürzen der Boeing 737-Max}

\href{https://www.nzz.ch/mobilitaet/luftfahrt/boeing-737-max-grundlegender-
softwarefehler-gefunden-ld.1499955}{Grundlegender Softwarefehler in der Boeing 737 Max gefunden}

\href{https://www.aerotelegraph.com/maengel-der-boeing-737-max-fuer-absturz-verantwortlich}{Mängel der Boeing 737 max für absturz verantwortlich}

\href{https://de.wikipedia.org/wiki/Boeing_737#Boeing_737_Max_8}{Boeing 737 max}

\href{https://www.boeing.com/737-max-updates/de/mcas/}{737 max updates}

\href{http://www.b737.org.uk/mcas.html/mantrim}{mantrim}

\href{https://deutsch.wikibrief.org/wiki/Boeing_737_MAX_certification}{Wikibrief - Boeing 737 MAX Certification}



\end{document}
