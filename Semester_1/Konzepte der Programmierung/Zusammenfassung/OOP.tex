\subsection{Objekt Orientierte Programmierung}

\subsection{Queues and Stacks}

\subsection{Priority Queue}

\subsection{Dictionaries and Binary Search Trees}

\subsection{Begleitobjekte}
\begin{itemize}
	\item Teile globale Variablen zwischen Exemplaren einer Klasse.
\end{itemize}
\begin{minted}[frame=lines, linenos, fontsize=\small]{scala}
object Studi:
	private var mattrikel_zaehler: Int = 0

class Studi(var name: String, var fach: String):
	var mattrikel_nr = Studi.mattrikel_zaehler
	Studi.mattrikel_zaehler = Studi.mattrikel_zaehler + 1
	def getMat(): Int = mattrikel_nr

@main
def test(): Unit = 
	var s1: Studi = new Studi("Max", "Info")
	var s2: Studi = new Studi("Katharina", "Binfo")
	var s3: Studi = new Studi("Günther", "Winfo")
	println(s1.getMat())
	println(s2.getMat())
	println(s3.getMat())
\end{minted}

\subsection{Vererbung}
\begin{itemize}
\item Soll Ziele der Wiederverwendbarkeit \& Erweiterbarkeit unterstützen.
\item Können Beziehungen zwischen Klassen herstellen
\item Können neue Klassen zu \textbf{Unterklassen} von bestehenden Klassen machen!
\item \textbf{Unterklasse} übernimmt damit automatisch alle Methoden und Attribute der \textbf{Oberklasse}
\end{itemize}
\begin{minted}[frame=lines, linenos, fontsize=\small]{scala}
class Person(var name: String, private var age: Int):
	def mature(): Unit = 
		age = age + 1
	def getAge(): Int = age
	def work(): Unit = 
		println("werkel werkel")
\end{minted}

\begin{itemize}
	\item Inklusionspolymorphie:
	\begin{itemize}
		\item Objekte der Unterklasse sind typekompotibel mit der Oberklasse
	\end{itemize}
	\item Wir unterscheiden:
\begin{itemize}
	\item \textbf{Statischer} Typ einer Variable:
	\begin{itemize}
		\item Typ aus Variablendeklaration (ändert sich nicht - statisch)
		\item legt fest auf welche Attribute und Methoden zugegriffen werden kann
	\end{itemize}
	\item \textbf{Dynamischen} Typ einer Variable:
	\begin{itemize}
		\item Typ des Objekts, auf das die Variable aktuell verweist
	\end{itemize}
\end{itemize}
\end{itemize}
Wir können den statischen Typ einer Variable mit einem Cast(Typumwandlung - nur wenn der dynamische Typ es erlaubt) ändern(asInstanceOf). Wenn er es nicht erlaubt, sonst bekommen wir eine Class:CastException als Error von Scala.

\subsection{späte Bindung/Überschreiben}
\begin{minted}[frame=lines, linenos, fontsize=\small]{scala}
class Studi(name: String, private var age: Int, var fach: String) extends Person(name, age):
	def getZurMensa(): Unit = println("Mjam, schlürf")
	override def work(): Unit =
		println("studier studier")
		
var s1: Studi = new Studi("Max", 12, "Data Science")
var s1: Studi = Studi@6870f51a

scala> s1.work()
studier studier

var p1: Person = s1
var p1: Person = Studi@6870f52a

scala> p1.work()
studier studier

# zweite person
scala> var p2: Person = new Person("Moritz", 10)
var p2: Person = Person@1107891a

scala> p1 = p2
p1: Person = Person@234234a

scala> p1.work()
werkle werkle

# beispiel mit Array von Personen wird gezeigt...das schreibe ich jetzt nicht ab
scala> var ps: Array[Person] = Array(p2, s1)

for p <- ps do
	p.work()
	
#ausgabe von scala
werkel werkel
studier studier
\end{minted}

\begin{tcolorbox}[colback=blue!5!white, colframe=blue!75!black, title=Beispiel-Titel, width=\textwidth]
Dies ist eine farbige Textbox mit einem Titel, einem blauen Hintergrund und einem Rahmen.
\end{tcolorbox}