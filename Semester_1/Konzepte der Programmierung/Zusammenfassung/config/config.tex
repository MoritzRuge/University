\usepackage[utf8]{inputenc}
\usepackage{xcolor} % Um Text Farbig markieren
%---Befehle-------------------------------------------------------------
% Einfach farbiger Text mit xcolor
%\textcolor{red}{Das ist roter Text.} \\
%\textcolor{blue}{Das ist blauer Text.} \\
%\textcolor{green!50!black}{Das ist dunkelgrüner Text.}
%---Befehle-------------------------------------------------------------
% Hintergrundfarbe für Text(\colorbox)
%\usepackage{xcolor}
%
%\begin{document}
%
%Normaler Text. 
%
%\colorbox{yellow}{Text mit gelbem Hintergrund.} \\
%\colorbox{cyan}{Text mit hellblauem Hintergrund.}
%-----------------------------------------------------------------------
\usepackage{tcolorbox}
\usepackage{blindtext} %erstellt lore ipsum
%---Befehle-------------------------------------------------------------
% \blindtext
%-----------------------------------------------------------------------


\usepackage{hyperref} % für links im Dokument
\usepackage{cleveref} % Intelligente Verweise
\usepackage{minted}%Paket für Codeblockdarstellung
\usepackage[backend=biber,style=numeric,url=true]{biblatex} % für Quellenangaben und Bibliotheken
\addbibresource{ref.bib}  % Add your .bib file here
\usepackage{graphicx} % Zum Einfügen von Bildern